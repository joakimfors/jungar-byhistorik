\jhsection{Skog, hemman Nr 1}

Före år 1831 hade hemmanet nr 32 i Nykarleby stads mantalslängd. Det gick då under namnet Skogsbyggare och nr 32 representerade dess plats i Överjeppo by och gränsade till Ytterjeppo by. Hemmanens antal längs i huvudsak älven i Överjeppo var 34 st, med Grötas som nr 1. Logiken i numreringen är för oss oklar, men fr.o.m. 1831 ändrar hemmanens mantalsnummer och Skogsbyggare får nu nr 1 och namnet ändras till enbart Skog. Hemmanet omfattade ½ mantal. Enligt släktforskarföreningen i Jakobstad innehades hemmanet mellan 1551-1563 av Nils Larsson, 1565-1607 av Mikkel Nilsson, 1609-1633 av Mårten Mikkelsson, 1623-1629 av Hans Hansson, 1629- ca.1653 av Carl Hindersson, 1657- 1673 av Matts Carlsson, 1675 av  Carl Mattsson, hu. Lisa Jönsdr., 1676-1709 av Tomas Mattsson och hu 1 Karin Hansdr, hu 2 Lisa, medåbo bror Markus Mattsson 1677-97, 1710-
1713 av sonen Jakob Tomasson och 1723- av Hans Hansson, hu Maria.

Efter Stora ofreden genomförde Matthias Wörman år 1740 en lantmäteriförrättning och kartläggning På hans kartblad finns endast en gård registrerad. 1783 års mantalslängd av Nykarleby stad upptar också endast en innehavare med 5/18 mtl. Små åkertäppor var  placerade i solskiftesform längs Romarbäcken. Det finns skäl att anta att Skogsbyggare (Skog) med sin placering en bit från älven var en bosättningsplats för backstugusittare under äldre tid. Från 1762 tilläts nämligen gifta legohjon att bygga backstugor och boningsrum på enskilda och samfällda ägor. Ändå har antalet bosatta på Skogsbyggare varit lågt. 1807 fanns antecknat endast 4 personer och hemmanet innehades av personer från Romar.
Utmärkande för Skog, i ännu högre grad än för övriga hemman, är den mycket stora emigrationen från hemmansnummern. Stora syskonskaror emigrerade i sin helhet till USA och emellanåt följde också föräldrarna efter. Idag finns endast en brukningsenhet kvar på hemmanet, som under hela 1900-talet hyste flera hemmansdelar. Däremot finns flera bostäder placerade på hemmanets mark.

\jhsubsection{Lägenheter på Skog}
