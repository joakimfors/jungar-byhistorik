
\jhchapter{Förord}

Historien kan inte tyglas. Det som har varit har varit. Minnen och hävder bleknar och kan återges på mångahanda sätt. Vi har valt att berätta byhistorien med start i nutid och därifrån steg för steg söka oss bakåt i varje enskild lägenhets tidslinje. Uppgiften har varit dryg, den har fordrat timmar och åter timmar av detekterande forskning, gårdsbesök, intervjuer, bild- och kartastudier och rotande i olika arkiv. Där det personliga minnet har räckt till, har berättelsen fått mera liv. I andra fall har historien givits en mera korthuggen form av årtal och konstateranden byggda på tillgängliga dokument.

Presentationen följer en hemmansvis ordning från norr till söder. Lägenhetens namn, registernummer, hemmanstillhörighet, plats på kartan samt husnummer anges allra först. Nuvarande ägare och personer som utgått från enskilda lägenheter har namngetts med personalia så noggrant det låtit sig göras och en berättelse kompletterar informationen. Där det handlar om hyres- eller s.k. genomgångsbostäder har utmaningen visat sig betydligt svårare att reda ut, även i nutid.

För läsaren kan upplägget till en början kännas ovant och svårt att greppa, men så småningom kommer logiken att sitta rätt. Med hjälp av kartorna och marginalnoteringarna jämte insatt nytt och gammalt bildmaterial, skall det gå att följa lägenheternas intressanta förflutna. Övriga samhällsfenomen i byn finns inbakade i texten och bidrar med sin krydda till helheten.

Arbetet har utförts på basen av kursen, ``Jungar bys lokalhistoria'', i Nykarleby Arbis under perioden hösten 2012 - sommaren 2017 och med Christer Fors som lärare. Till en början var deltagargruppen större än den var i slutändan, men den efterfikade kunskapen och informationen har med envis ihärdighet lyckats fås fram. Under jobbet med datainsamlingen har dessutom s.g.s. varje hushåll i byn getts möjlighet att bidra med sin andel. Allt material har inte kunnat ges utrymme i denna bok. Däremot har det sparats för eventuellt kommande bruk.

Kursdeltagarna har varit:
\begin{center}
  \begin{tabular}{l l l l}
    Christer Fors & ledning, data, skribent & Dorita Jungarå & datainsamling \\
    Lea Stenvall & data, skribent & Olav Jungarå & data, skribent \\
    Gunnel Elenius & data, skribent & Paul Björkqvist & data, skribent \\
    Fjalar Fors & data, skrib., redigering & Lars Silfvast & data \\
    Greta Back & data & Paul Laxén & data, skribent \\
    Leif Forss & data, skribent & Ingeborg Forss & data \\
    Mayvor Fors & foto, kartor & Hans Kronlund & medlem \\
    Johannes Forss & data & Bruno Strengell & data \\
    Carl-Erik Forss & medlem & Ing-Britt Forss & data \\
    Rolf Gunnar & data & vem & medlem \\
  \end{tabular}
\end{center}
Utöver dessa har Joakim Fors i Dalby, Lund, på distans lagt upp, övervakat och säkrat redigeringsprocessen för detta verk. Redigeringsarbetet påbörjades 20 april 2017 och avslutades ........... 2017.
