
\jhchapter{Förord}

Historien kan inte tyglas, tiden visar ingen hänsyn. Det som har varit är borta och vi kan på sin höjd ana hur fenomenen i all hast dragit förbi. De minnen vi fångat är skärvor från en gemensam bakgrund som kan återges på många olika sätt. Vi har valt att berätta byhistorien med START I NUTID och därifrån steg för steg söka oss bakåt i varje enskild lägenhets tidslinje. Vi har delvis använt boken ``Överjeppo 2'' som modell. Uppgiften har varit dryg, den har fordrat timmar och åter timmar av detekterande spårning, gårdsbesök, intervjuer, bild- och kartastudier och rotande i många arkiv. Där det personliga minnet har räckt till, har berättelsen fått extra liv. I andra fall har historien givits en mera korthuggen form av årtal och konstateranden byggda på tillgängliga dokument.

Detta är i huvudsak fastigheternas historia och presentationen följer en hemmansvis ordning från norr till söder. Lägenhetens namn, registernummer, hemmanstillhörighet och plats på kartan med eget husnummer anges allra först. Nuvarande ägare och personer, som utgått från respektive nummer, har namngetts med personalia så noggrant det låtit sig göras och en berättelse kompletterar vanligtvis informationen. Där det handlar om hyres- eller s.k. genomgångsbostäder har utmaningen visat sig betydligt svårare att hantera då register eller hyresmatriklar inte funnits tillgängliga.

För läsaren kan upplägget till en början kännas ovant och svårt att greppa, men så småningom kommer logiken att sitta rätt. Med hjälp av kartorna och marginalnoteringarna jämte insatt nytt och gammalt bildmaterial, skall det gå att följa lägenheternas intressanta förflutna. Övriga samhällsfenomen i byn finns inbakade i texten och bidrar med sin krydda till helheten.

De obesuttna, i form av torpare och backstugusittare, dyker här och där upp i materialet, men utgör tillsammans med pigor, drängar, inhyseshjon och s.k. ``lösa personer'' en så stor kategori att en samlad berättelse om dem, på grund av utrymmesbrist, kanske får bli ett nytt projekt som ytterligare berikar historien om Jungar by.

Arbetet har utförts på basen av kursen, ``Jungar bys lokalhistoria'', i Nykarleby Arbis under perioden hösten 2012 -- sommaren 2017 och med Christer Fors som mentor. Till en början var deltagargruppen större än den var i slutändan, men den efterfikade kunskapen och informationen har man med envis ihärdighet lyckats få fram. Under jobbet med datainsamlingen har dessutom så gott som varje hushåll i byn getts möjlighet att bidra med sin andel. Med detta sagt har allt material inte kunnat ges utrymme i denna bok. Däremot finns det sparat för eventuellt kommande bruk.

Deltagarna har varit:

\begin{center}
  \begin{tabular}{l l l l}
    Christer Fors & ledning, fakta, skribent & Dorita Jungarå & faktainsamling \\
    Lea Stenvall & faktainsamling, skribent & Olav Jungarå & fakta, skribent \\
    Gunnel Elenius & fakta, skribent & Paul Björkqvist & fakta, skribent \\
    Fjalar Fors & fakta, skrib., redigering & Lars Silfvast & fakta \\
    Greta Back & fakta & Paul Laxén & fakta, skribent \\
    Leif Forss & fakta, skribent & Ingeborg Forss & deltagare \\
    Mayvor Fors & fotografier, kartor & Hans Kronlund & deltagare \\
    Johannes Forss & fakta & Bruno Strengell & fakta \\
    Carl-Erik Forss & deltagare & Ing-Britt Forss & fakta \\
    Rolf Gunnar & deltagare &  &  \\
  \end{tabular}
\end{center}

Utöver dessa har Joakim Fors i Dalby, Lund, på distans lagt upp, övervakat och säkrat den bakomliggande redigeringsfunktionen i \LaTeX för detta verk. Daniel Fors har bistått med teknisk support på hemmafronten. Redigeringsarbetet påbörjades den 20 april 2017 och avslutades till huvudsakliga delar den 27 augusti 2017.

Utgivningen av boken stöds med bidrag från Svenska kulturfonden, xy, zå och äö.


Christer Fors
