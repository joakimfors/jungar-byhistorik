
KARTBLAD nr 6 placeras hit --------------->>>


\jhhouse{Sparbanken 1}{4:194}{Silvast}{6}{81}


\jhoccupant{Fastighets Ab}{Sparbanken Jeppo}{2001 -}

\jhbold{Från by med bank till banklös by}

\jhbold{Jeppo Sparbank	-   Vasa Andelsbank   -	Fastighets Ab Sparbanken Jeppo}

Redan före kriget diskuterade de ekonomiskt tänkande jeppoborna behovet av en egen bank, en sparbank eller en andelskassa, utöver det dåvarande och redan mångåriga filialkontoret Helsingfors Aktiebank. Ofärdsåren gjorde emellertid att planerna tog form först år 1946, då man vid ett talrikt besökt ÖSP-möte den 9 maj tog frågan till diskussion. Beslutet blev att tillsätta en kommitté för att förbereda ärendet och föra det vidare.

En månad senare sammankallades till allmänt möte på Uf-lokalen. Solidariteten till Helsingfors Aktiebank var påtaglig, men en majoritet fick mötet att efter noggrant övervägande besluta anhålla hos kommunalnämnden att vidta åtgärder för bildande av en sparbank i Jeppo. Anhållan är daterad 27.6.1946, signerad av Edvin Bergman, Joel Sandberg, Ernst Westerlund, Haniel Sandberg och Lennart Jungar, därtill stödd av Jeppo Andelsmejeri och Jeppo Skogsandelslag. Den fick grönt ljus i såväl kommunalnämnd som i ett enhälligt -fullmäktige, som i sitt protokoll den 11.9.1946 skrev: ``Beslöts att stifta en sparbank i kommunen och anslogs för densamma såsom grundfond 50.000 mk.''

Kvarnarna malde snabbt. Den 29.10.1946 godkände kommunalfullmäktige förslaget till stadgar och den 20.12.1946 konstaterade samma församling att dessa godkänts av Finansministeriet. Vid det nämnda decembermötet utsågs de första principalerna.

Bankens första lokalitet blev från 10.2.1947, för en månadshyra på 600 mk, Johannes Lindéns gård i centrum (senare Jeppo Sko- och Läderaffär /Runar Källström, Kirsiläs skomakeriverkstad fram till 1.8.1983, etc.) på nuvarande Östra Jeppovägen 19 (se nr 78). Storstugan i söder  inreddes till banklokal med behövlig lång disk, hyllor, stolar och kassaskåp. Lokalen var otrevligt kall. Hösten 1948 flyttade verksamheten in i ett varmare kaférum i Jeppo-Oravais Handelslag. Handelslaget ansåg sig redan i början av 1950 behöva utrymmet för egen verksamhet, varför banken blev tvungen att hastigt hitta en annan lokal.

På den tomt där banken nu finns, stod en gård (se nr 381) tillhörande Helli Sarelin, som i rummet mot vägen drev en kemikaliebutik. Med henne kom styrelsen överens om att hyra de övriga rummen mot gårdssidan, på villkor att fru Sarelin fick fortsätta med sin rörelse. Senare samma vår beslöt vårmötet att köpa hela fastigheten om 0,0007 mtl, benämnd Nygård lägenhet RN:o 4:9 av Silvast hemman, för löseskillingen 850.000 mk. I den vevan uppmanades Sarelins handel med kemikalier att flytta ut med 2 veckors varsel.

Vid höstmötet 1950 beslöts bryta ut tomten och vid tillfälle utvidga den mot norr, fram till ungdomslokalens tomt. Utvidgningen kunde verkställas i samband med höstmötet 1953. Då godkändes inköpet av en 800 m2 tilläggstomt till priset 170.000 mk. Därmed drogs rån i norr till den plats där den nu finns.

Under detta årtionde kom bankkunderna ofta med häst och behövde möjligheter att binda fast sitt ekipage medan de utförde sitt ärende. Därför anskaffades skyndsamt en stadig bom för ändamålet. I dagens värld har denna bytts mot asfalterade och uppskottade parkeringsfickor för bilar.
Med tiden utökade banken sin verksamhet med nya serviceformer och mera personal. Trängseln blev olidlig och styrelsen började sondera möjligheterna att bygga nytt på den befintliga tomten. År 1958 företogs en studieresa till Sydösterbotten. Den gjorde styrelsen klar över att ett nybygge låg väl inom möjligheternas gräns. Då även kommunen visade sig intresserad av hyreslokaler i ett planerat bankbygge, var det rätt tryggt att gå från tanke till handling. Styrelsemötet den 26.2.1959 och en senare besiktning under vårvintern kom fram till att markgrunden var säker. Man kunde därmed släppa planen på ett enplanshus i trä och gå in för en byggnad av sten i två våningar.

I april 1959 antog styrelsen bland tre inlämnade förslag den ritning som K-E Nyman uppgjort. Efter smärre ändringar godkändes dessa tillsammans med arbetsbeskrivningen den 11.5.1959. Nu kunde man utlysa byggprojektet på entreprenad. Flera anbud inkom. Den 13 juni avgjorde styrelsen dessa på följande sätt:

\begin{center}
  \begin{tabular}{l l r}
    \hline
    Uppdrag & Ansvarig & Kostnad \\ \hline
    Byggande & Georg Romar, byggm. & 11.443.000 mk \\
    Rör- och vattenanläggning & Johannes Fors & 1.147.000 mk \\
    Elinstallationer & Jeppo Kraft Alg & 448.000 mk \\
    Jordkabel & Jeppo Kraft Alg & 55.000 mk \\
    Ritningar & K-E Nyman & 200.000 mk \\
    Arbetsövervakning & o & 50.000 mk \\
    \hline
  \end{tabular}
\end{center}

Målaren Olov Laxén vidtalades att göra förslag till färgsättning.

För att säkerställa finansieringen upptogs ett 7 milj.mk stort lån hos SCAB. Efter ett ovanligt raskt byggande var huset inflyttningsklart redan den 11.2.1960. För 15 milj.mk, vilket var satt som kostnadstak, hade man fått ett modernt hus med adekvata bankutrymmen och sidorum, mottagningsrum för hälsovården, tre boningslokaler i övre våningen avsedda för bankdirektör, hälsosyster och barnmorska samt i källaren klubblokal, arkivrum och kök. Invigningsfest hölls den 15.10.1960.


\jhhousepic{037-05573.jpg}{}

TILLBYGGNAD

Erfarenheten efter sex års användning visade på vissa brister. Man klagade på att klubbrummet i källaren var för litet, mörkt och dåligt ventilerat. Styrelsen väckte därför hösten 1966 förslag om tillbyggnad av ett större klubbrum i söder, med garage i bottenvåningen. Våren 1967 gav principalerna klartecken för tillbyggnad av 120 m2. Med Gunnar Kronqvists ritningar fullföljdes byggandet i egen regi med Georg Romar som ansvarig ledare. Totala kostnaden för bygget, med fullständig inredning, blev 65.000 mk. Resultatet utföll i en stor och ljus klubblokal, som fram till hösten 2014 varit mycket uppskattad och i flitig användning av föreningar o.a. Till bybornas besvikelse sattes den nu i annan funktion.


FRÅN SPARBANK TILL ANDELSBANK OCH INGEN BANK

År 1991, den 9.10, anslöt sig Jeppo Sparbank till Sparbanken Deposita, som uppstått 1966 som ett resultat av en fusion av sparbankerna i Jakobstad och Nykarleby. Redan 31 januari 1992 anslöt sig Deposita av solidaritetsskäl till Sparbanken i Finland i ett försök att rädda den krisande sparbanksrörelsen. Det hjälpte inte. Kreditförluster och flykten från banken gjorde att statsrådet den 22.10.1993 beslöt sälja Sparbanken i Finland i fyra lika stora delar till de övriga bankgrupperna i Finland, d.v.s. till Andelsbankerna, Föreningsbanken, Postbanken och KPO.

Redan följande morgon då bankfolket kom till jobbet hörde de telefaxerna knattra fram den av Finlands Bank uppgjorda fördelningen, som visade att kontoret i Jeppo tillförts Vasa Andelsbank. Endast 1,5 timme senare stegade den nya vd:n Kaj Skåtar in i lokalen för att beskåda sitt nyförvärv och lugna personalen för vilken ``golvet plötsligt gett vika under fötterna''. Nu skulle skräplånen, som uppgick till ca 40 miljarder mark inom sparbanksgruppen, överföras till den nygrundade banken för detta ändamål, Arsenal.

Jeppo-kontoret var filial under VAB och hyresgäst i fastigheten tills det drygt 20 år senare stängdes i juni 2014 p.g.a. rationaliseringsåtgärder inom andelslaget. Stängningen möttes av omfattande men resultatlösa protester från byinvånarna. F.o.m. nu saknas ett fysiskt bankkontor på orten!

Sedan år 2001 äger bolaget Fastighets Ab Sparbanken Jeppo (Greger Nygård) bostads- och affärsfastigheten Sparbanken 1, R:no 4:194 i Jungar by.


DIREKTÖRER

Skötseln av bankens verksamhet anförtros verkställande direktören. Vanligen är det också den personen som är bankens ansikte utåt. Att man det stora ansvaret till trots har trivts på sin post visas av att enbart sex personer hunnit bekläda den:
\begin{enumerate}
  \item Rakel Törnkvist (Romar),	1947
  \item Lisbeth Ågren (Jungar),	1947-1951
  \item Gustav Johansson,		1952-1965
  \item Johan Stenfors,		1965-jan. 1993
  \item Marit Leinonen,		1993-apr. 1994
  \item Ing-Britt Palm (Fors),	1994-2004
  \item Ralf Bonde, 			2005- --> juni 2014, enhetschef fram till stängningen
\end{enumerate}

INHYSTA VERKSAMHETER

Många verksamheter, av vilka några nämndes ovan, har tidvis varit inhyrda i husets markplan. Hösten 1968 flyttade Nykarleby-Jeppo Brandförsäkringsförening(försäkringsförening), ÖSP:s skattetjänst och Jeppo Skogsandelslag/Jeppo Skogsvårdsförening in. Två olika frisörsalonger, bokföringsbyrå liksom en nybliven kosmetolog har här funnit en lämplig verksamhetsadress.


HYRESGÄSTER/BOENDE

DIREKTÖRSBOSTADEN:
\begin{enumerate}
  \item Gustav och Gun-Britt Johansson samt barnen Bo, Ulf och Susanne, 1960-nov 1965
  \item Johan och Barbro Stenfors, barnen Ann-Marie och Krister, nov 1965-dec 1977
  \item Sven och Christina Simons samt barnen Peter och Jan, dec 1977-dec 1981
  \item ???, 1982-
  \item m
  \item m
  \item m
\end{enumerate}
NORRA ÄNDAN:
\begin{enumerate}
  \item Hemming och Lisbeth Pantolin samt dottern Maria, 1960-nov.1968
  \item Sven och Christina Simons,  barnen Peter och Jan, \textborn 2.3.1975
  \item Tage och Ann-Christine Fors, dec 1977-?
  \item m
  \item m
  \item m
  \item m
\end{enumerate}
MELLERSTA LOKALEN:
\begin{enumerate}
  \item Bernhard och Thea Fogström m. barnen Britt-Mari, Anders och ?, 1960-nov 1968
  \item Caj och Linnéa Fagerlund samt dottern Carita, nov 1968-sommaren 1972
  \item Helga Enlund, sommaren 1972-?
  \item Björn och Ing-Britt Palm, 19??
  \item m
  \item m
  \item m
  \item m
  \item Johan och Emma Sjölind, 20---2017
\end{enumerate}



\jhhousepic{Sparbanksvy 50-talet.jpg}{Sparbanken innan nybygget tog dess plats. Gaveln på ``Jeppoboden'' t.v.}


\jhhousepic{Sparbanken ca 1950}{Tidigare Helli Sarelins kemikalieaffär}


\jhhouse{Nygård}{4:9}{Silvast}{6}{381}


\jhoccupant{Jeppo}{Sparbank}{1950-1960}

Banken kom att använda fastigheten under 10 års tid innan den förverkligade planerna på nybygge och kunde riva det gamla huset.	Boende åren
1957 – 1960: Gustav och Gun-Britt Johansson samt barnen Bo, Ulf och Susanne.

\jhoccupant{Helli}{Sarelin}{19..-1950}

Fru Helli Sarelin innehade kemikalieaffär i denna fastighet i ett rum ut mot vägen. Resten av huset användes som bostad. I början av 	1950 hyrde hon bostadsrummen till den då husvilla sparbanken mot löfte om att kunna fortsätta sin rörelse i butiken. När hon samma vår gick med på att sälja hela fastigheten, fick hon endast två veckor på 	sig att flytta ut.


Källor:
Jeppo Sparbank 1947-1977, sammanställd av Runar Nyholm
Ralf Bonde
Sökmotorn Google


\jhhouse{Verkstaden}{3:70}{Fors}{6}{94}


\jhoccupant{Sundqvist}{Anders \& Ann-Kristin}{1997 -}
Erik Anders Erland, \textborn 26.08.1955 i Nykarleby, gifte sig den 28.12.1991 med Märta Ann-Kristin, \textborn 14.11.1961, född Björk från Karleby, Såka.
\begin{jhchildren}
  \item \jhperson{Emma Kerstin Marita}{13.02.1993}{}, vårdbiträde
  \item \jhperson{Erik Anders}{02.08.1995}{}, vvs-montör
\end{jhchildren}

Innan familjen slog sig ner i Jeppo bodde Anders och Ann-Kristin i Nykarleby. Anders är lagerarbetare vid KWH Mirka i Jeppo och Ann-	Kristin har bl.a. arbetat som projektledare vid Optima samkommun i 	Jakobstad innan hon övergick till funktionen som verksamhetsledare 	på Psykosociala föreningen Contact rf på samma ort.

\jhhousepic{036-05571.jpg}{}

Husets bottenvåning är numera inredd för boende med bl.a. sovrum, bastu och tvättstuga. Entrén har byggts om till inglasat uterum.


\jhoccupant{Björkqvist}{Stig \& Camilla}{1995-1997}

Stig-Erik Ivar, \textborn 11.02.1960, gifte sig 20.06.1992 med Camilla Ann-Christine Strengell, \textborn 11.12.1971 i Jeppo.
\begin{jhchildren}
  \item \jhperson{Emilia Josefine}{04.10.1988}{}, stud. i Aberdeen
  \item \jhperson{Oscar Ivar Jonathan}{22.01.1996}{}
  \item \jhperson{Filip Erik Cesar}{23.04.2002}{}
  \item \jhperson{Alexander Paul Lucas}{23.04.2002}{}
\end{jhchildren}

Efter flytten från Jeppo har familjen bott i Esbo och Maastricht. Nu finns bopålarna i Kyrkslätt. Stig jobbar som Business controller för 	Rettig Group och Camilla som barnträdgårdslärare för Kyrkslätt stad.

\jhoccupant{Lindén}{Gurli}{1988-1995}

Se nedan! Efter separationen från Boris blev Gurli och Eva Stina Byggmästar	boende i huset. Gurli fortsatte sitt etablerade och framgångsrika författarskap. Under den här tiden utkom hon med:
	- 1989	``Framtid''	Författarnas Andelslag
	- 1990	``Pyramid''	   ''		     ''
	- 1993	``Grodan''	   ''		     ''

Andelslagets verksamhet var under en stor del av 70- och 80-talet stationerad i huset. Eva Stina inledde här sitt författarskap.

\jhoccupant{Lindén}{Boris \& Gurli}{1962-1988}

Boris Runar, \textborn 07.06.1940 i Jeppo, gifte sig den 02.04.1961 med Gurli Ingegärd Kronqvist, \textborn 18.01.1940, från Öja.
\begin{jhchildren}
  \item \jhperson{Thomas}{01.09.1961}{}
  \item \jhperson{Andreas}{08.05.1964}{}
  \item \jhperson{Johanna}{09.06.1967}{}
\end{jhchildren}

Boris utnyttjade verkstaden i huvudsak som bilverkstad. Gurli hade syateljé i hemmet under 60-talet. Senare, år 1970, flyttade 	verkstadsverksamheten till Boris' nybyggda bensinstation och verkstad Union (Grötas nr 11), där också annat nytänkande tog form och där man får anta att ett frö till företaget BSB Mekan grodde.


\jhbold{På hyra eller arrende}

\jhperson{\jhbold{Enlund}}{Valdemar \& Berit}{1955-1960}

Anders Lennart Valdemar, \textborn 27.09.1925 i Kengo, gifte sig den 08.07.1950 med Berit Maria Elenius, \textborn 02.09.1926 i Jungar, Jeppo.
\begin{jhchildren}
  \item \jhperson{Lisen}{31.05.1951}{}, bor i Terjärv
  \item \jhperson{Britten}{04.12.1956}{}, bor i Jeppo, se Fors nr 131
  \item \jhperson{Anders}{30.11.1966}{}, bor i Närpes
\end{jhchildren}

Valdemar och Berit, som bodde på Holmen, hyrde gården av Vilhelm 	Björklund. Tillsammans fortsatte Valdemar och hans bror Bror med 	samma typ av arbeten som redan förekommit i verkstaden. År 1960, då Valdemar fick anställning som bilskollärare vid Haldin \& Rose i Jakobstad, flyttade familjen bort från orten.

\jhoccupant{Björklund}{Vilhelm \& Elsa}{1954-1962}
Jarl Vilhelm, \textborn 17.07.1906 i Jeppo, gifte sig den 19.08.1934 med Elsa Erika Lindström, \textborn 06.10.1912 i Vasa, från Jeppo.
\begin{jhchildren}
  \item \jhperson{Karin}{1936}{}
  \item \jhperson{Gustav}{1942}{}
\end{jhchildren}

Den 8.9.1954 köpte Vilhelm den 600 m2 stora tomten från lägenheten Nybonde 3:56 för sin verkstad. I verkstaden utförde han ensam all slags reparationer på traktorer, lantbruksmaskiner och bilar, eller enkelt sagt, allt det som folk kom med som skulle åtgärdas. Vilhelm saknade sin ena hand efter en olycka som barn, den var 	endast en liten klump längst ut på armen, men han behärskade ändå sitt hantverk mycket skickligt.

Ett utrymme i bostaden användes som frisörsalong för Elsa. Familjen flyttade så småningom till Sverige, där Vilhelm och Elsa också slutade sina dagar.

Vilhelm \textdied 1982  --  Elsa \textdied 2007

\jhoccupant{Westerlund}{Ernst \& Agnes}{1945-1954}

Ernst Eric, \textborn 15.11.1913 i Jeppo, gifte sig den 10.10.1939 med Agnes Emilia, \textborn 21.01.1914, född Lindfors på Silvast (se 359).
\begin{jhchildren}
  \item \jhperson{Greta}{02.09.1942}{}
  \item \jhperson{Gunnevi}{27.08.1947}{}
  \item \jhperson{Gun-Britt}{11.04.1949}{}
\end{jhchildren}

Ernst och Agnes övertog denna del av lägenheten Nybonde 3:56 den 6.2.1945, men torde knappast ha bott i det nybyggda huset, eftersom 	de redan då hunnit etablera sig på Timmerbackholmen (karta 9 nr 127). Därför arrenderade de ut och senare sålde det till Vilhelm Björklund för hans ändamål.

\jhoccupant{Westerlund}{Eric \& Amanda}{1938-1945}

Eric Jakobsson, \textborn 18.05.1888, gifte sig den 02.02.1913 med Amanda Irene Jakobsdotter Nybonde, \textborn 29.07.1892 i Soklot.
\begin{jhchildren}
  \item \jhperson{\jhbold{Ernst Eric}}{15.11.1913}{13.12.1992}, i USA
  \item \jhperson{Bruno Johannes}{26.06.1919}{}
  \item \jhperson{Edna Amanda}{08.06.1922}{}, gift Eklöv
\end{jhchildren}

Eric och Amanda byggde huset åren 1937-38. Övre våningen byggdes som bostad medan markvåningen, uppförd i rött tegel, inreddes för 	företagsändamål. Det sägs att man under kriget tillverkade potatismjöl i husets  bottenvåning. Då Eric dog redan 13.10.1940, flyttade Amanda tillbaka till den ursprungliga bostaden (nr 396).

\jhhousepic{Amanda W.jpg}{Bild: Sv. Österbottens bebyggelse i ord och bild, Borås 1965}

Amanda \textdied 07.07.1984


\jhhouse{Dahllund}{3:42}{Fors}{6}{396}

\jhoccupant{Jeppo Kraft}{Alg}{}


\jhoccupant{Jeppo Sparbank}{Arsenal?}{19--}


\jhoccupant{Dödsbo}{Westerlund}{1984-19--}


\jhoccupant{Westerlund}{Eric \& Amanda}{1912-1984}

Eric Jakobsson, \textborn 18.05.1888,  gifte sig den 02.02.1913 med Amanda Irene Jakobsdotter Nybonde, \textborn 29.07.1892 i Soklot. Jfr ovan med Fors nr 94.
\begin{jhchildren}
  \item \jhperson{\jhbold{Ernst Eric}}{15.11.1913}{13.12.1992}, i USA
  \item \jhperson{Bruno Johannes}{26.06.1919}{}
  \item \jhperson{Edna Amanda}{08.06.1922}{}, gift Eklöv
\end{jhchildren}

Eric beskrivs som bonde på Fors på del av faderns hemman (5/96 mantal), som övertogs 1912 från Jakob och h h Kristina Westerlund. Eric och Amanda uppförde bostadshuset med två rum och kök år 1913 i trä under tegeltak. Samma år byggdes gårdsbyggnaden i trä under pärttak. Mellan bostaden och uthuset grävdes en vattenbrunn som gav bra dricksvatten ännu in på 70-talet, vilket de närmaste grannarna med tacksamhet vindade upp och utnyttjade.

Nya företagsplaner slog så småningom rot och åren 1937-38 lät paret bygga ett nytt hus på angränsande tomt (nr 94, Verkstaden 3:70). Då Eric dog en kort tid efteråt, flyttade Amanda tillbaka till den ursprungliga bostaden.

Eric \textdied 13.10.1940  --  Amanda \textdied 07.07.1984.


\jhoccupant{Westerlund}{Jakob \& Kristina}{1910-1912}

Jakob Westerlund, \textborn 25.10.1850 på Biggas, son till Jan Jakobsson Biggas, gift 11.10.1874 m. Kristina Eriksdotter Pet, \textborn 04.11.1854 på Pet.
\begin{jhchildren}
  \item \jhperson{Joel}{1875}{}, till Amerika
  \item \jhperson{Viktor}{1877}{}, till Nykarleby 1897
  \item \jhperson{Johannes}{1880}{}, till Amerika
  \item \jhperson{Hanna}{1883}{1951}, gift Sjöblom
  \item \jhperson{Wilhelm}{1886}{}, till Amerika, gift med jeppoflicka (1 barn)
  \item \jhperson{\jhbold{Eric}}{1888}{}, bonde på Fors (5/96 mantal)
  \item \jhperson{Emil}{1890}{}
  \item \jhperson{Ingrid Maria}{1892}{}
  \item \jhperson{Gustav}{1894}{}
  \item \jhperson{Ture Evert}{1897}{1952}, taxi, limonadtillverkning i Silvast
\end{jhchildren}

Jakob var byskollärare i Korsholm 1870-72. Familjen flyttade via Nykarleby till Stenbacken år 1886/-88, där de bodde fram till 1910, då de flyttade till Silvast. Jakob tjänstgjorde som landspolis med många förtroendeuppdrag i kommun och församling. Ägare till en del av Fors skattehemman Rno 3:8; svårt att veta var de bodde under dessa två år.

Jakob \textdied 08.03.1914  -- 	Kristina \textdied 14.01.1936
