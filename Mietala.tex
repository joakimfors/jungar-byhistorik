\jhchapter{Mietala, hemman Nr 9}

\jhbold{Mietala, hemman  Nr 9}

textpresentation


Mietala hemman omfattas av vidstående karta nr \jhbold{14}.


KARTA nr 14 hit --->


\jhsubsection{Lägenheter på Mietala}


\jhhouse{Paloåkern}{9:91}{Mietala}{14}{1, 1a-c}

\jhoccupant{Forsgård}{Kjell \& Annette}{1991 -}
Kjell, \textborn 15.08.1966 i Jeppo, gifte sig 1994 med Annette  Nilsson, \textborn 24.11.1969 från Kronoby. Kjell är agrolog till utbildningen. Han övertog föräldrarnas lägenhet på Mietala år 1991. Annette är tandhygeinist och har under åren arbetat deltid vid sidan av jordbruket, f.t arbetar hon på Tandklinik Hedman/Nygård i Nykarleby.

\jhhousepic{216-05794.jpg}{Kjell och Anette Forsgård}

\begin{jhchildren}
  \item \jhperson{Fanny}{15.01.1997}{}
  \item \jhperson{Felix}{15.07.1998}{}
  \item \jhperson{Filippa}{21.08.2002}{}
\end{jhchildren}
Lägenheten består av 73 ha åkermark och 13 ha skog. Huvudnäring har varit potatisodling och kalvuppfödning - sistnämnda avslutades 2017. Yngve Forsgård byggde bostadshuset år 1963 och fähuset 1964. Bostadshuset förstorades och renoverades år 1996 av Kjell och Annette. År 1993 byggde Kjell en byggnad för kalvlösdrift, år 1997 med pappa Yngves hjälp en potatiskällare, år 2011 en maskinhall.

Kjell har varit aktiv inom olika föreningar, bl.a har han varit ordförande i ÖSP:s Jeppo-avdelning, ledamot i fullmäktige  samt ordförande för idrottsföreningen. Kjells helhjärtade insatser för byggandet av Måtarsstugan bör också nämnas.


\jhoccupant{Forsgård}{Yngve \& Berit}{1963-1991}
Yngve, \textborn 06.01.1933 i Jeppo, gift med Berit Slangar, \textborn 21.12.1939 på Slangar i Jeppo.
\begin{jhchildren}
  \item \jhperson{\jhbold{Kjell}}{15.08.1966}{}
  \item \jhperson{Monica}{28.11.1970}{}, socionom, gift Eklund
\end{jhchildren}
Berit hade tjänst på Jeppo Sparbank före giftermålet. Hon och Yngve övertog en tredjedel av hans föräldrars lägenhet. Senare köpte de till odlad jord för sitt jordbruk. Yngve byggde egnahemshuset och fähus på Paloåkern. Berit har bl.a varit principal i sparbanken, kassör i Pensionärs- och Marthaföreningen samt med i Hembygdsföreningens och Jeppo ungdomsförenings styrelse.

Till Yngves fritidsintresse hörde musiken. Som ung sjöng och spelade han i flera orkestrar, bl.a med sina bröder i orkestern Rythm Boys. Musikaliska uppdrag följde honom sedan livet ut. I flere år var han ledare för pensionärsföreningens sångare.

Yngve och Berit flyttade till radhuslägenhet vid Åkervägen 7 i samband med generationsväxlingen.
Yngve \textdied 10.10.2013.



\jhhouse{Riåkern}{9:109}{Mietala}{14}{2, 2a}

\jhoccupant{Kalijärvi \& Englund}{Tony \& Janina}{2016 -}
Tony Kalijärvi, \textborn 15.04.1990 i Jeppo och Janina Englund, \textborn 11.01.1992 i Jakobstad, köpte gården år 2016. Tony arbetar på Mirka i Jeppo och Janina arbetar på Sale i Bennäs.

\jhhousepic{218-05901.jpg}{Tony Kalijärvi och Janina Englund}

\jhoccupant{Norrgård}{Olof \& Dagmar}{1978-2016}
Olof Norrgård, \textborn 24.10.1935 på Mietala, gift med Dagmar Sundell, \textborn *02.05.1935  i Jeppo. Olof har arbetat som minkfarmare och chaufför, Dagmar som butiksföreståndare och kontorist. En ny bostadsbyggnad uppfördes år 1978 av Oravais Hus och paret flyttade in. Garaget uppfördes år 1979 av Evert Norrgård, Olofs far.

Olofs fritidsintresse var musik, han spelade som ung i Jeppo hornorkester.

Olof \textdied 30.04.2014  --  Dagmar \textdied 31.01.2016



\jhhouse{Kvarnfors}{9:77}{Mietala}{14}{102}

\jhoccupant{Norrgård}{Evert \& Signe}{1951-1990}
Evert Norrgård, \textborn 01.02.1905, gifte sig 6.11.1932 med Signe Gunell, \textborn 2.4.1906 i USA. Evert var byggnadssnickare och chaufför. Han har byggt ett tiotal egnahemshus i Jeppo. På 1950-talet startade han tillsammans med Karl Lindgren den \jhbold{första privata minkfarmen i Jeppo}, senare fanns också en tredje delägare, nämligen H Bäckman, farmchef på Keppo. Bäckman emigrerade senare till Australien. Farmen avslutades 1971 pga dåliga skinnpriser och Evert blev pensionär.
\jhhousepic{Mietala102-NorrgardEv.jpg}{Evert och Signe Norrgård}

Evert var en naturvän, som sökte ovanliga blommor och växter i åbacken. Han kom ofta in med den första konvaljbuketten på våren. Signe var med i Marthaföreningen. Hon städade vid Gunnar småskola under ett par år.
\begin{jhchildren}
  \item \jhperson{Ingmar}{08.06.1933}{}, tekniker, bor i Sverige
  \item \jhperson{\jhbold{Olof}}{24.10.1935}{}, chaufför
  \item \jhperson{Börje}{24.11.1938}{02.08.2008}, tekniker, Smedsby
  \item \jhperson{Marléne}{03.12.1943}{}, merkonom, bor i Larsmo
\end{jhchildren}
Evert byggde bostadshuset år 1951 samt ekonomibyggnaden 1954. Boningshuset uppfört i spirvirke med sågspån som isolering. Huset revs i januari 2010. Lägenheten har tillhört samma släkt sedan 1800-talet och övertogs av Evert efter fadern år 1923.

Signe \textdied 23.06.1988	--	Evert \textdied 31.07.1990



\jhhouse{Norrgård}{9:78}{Mietala}{14}{3, 3a}

\jhoccupant{Norrgård}{Bo \& Marita}{2014 -}
Ägare till denna gård samt tomt är sedan år 2014 Bo och Marita Norrgård. Se närmare Mietala gård nr 5.


\jhoccupant{Norrgård}{Rune}{1973-2014}
Rune Norrgård, \textborn 03.09.1949 på Mietala. Rune blev student från Nykarleby Samskola 1970 och studerade därefter språk och teologi vid Åbo akademi. År 1973 övergick hemmanet i hans namn. Han blev dock sjukpensionerad pga ohälsa och bodde ensam i hemgården till sin död, 27.01.2014.
\jhhousepic{219-05795.jpg}{Rune Norrgård fram till 2014, nu Bo och Marita}

\jhoccupant{Norrgård}{Lennart \& Edit}{1900-1973}
Lennart Norrgård, \textborn 18.06.1908 på Mietala, gift med Edit Backlund,	\textborn 19.03.1910 på Måtar.
\begin{jhchildren}
  \item \jhperson{Erik}{27.05.1936}{}, (Silvast 355 )
  \item \jhperson{Göran}{23.08.1941}{}, bilmekaniker, bor i Sverige
  \item \jhperson{Bo}{19.03.1945}{}, (Mietala 5)
  \item \jhperson{\jhbold{Rune}}{03.09.1949}{}
\end{jhchildren}
Edit och Lennart var småbrukare på ena halvan av hans fars lägenhet. Lennarts yrkesverksamma liv präglades dock mer av de arbeten, som han genom sitt tekniska intresse och sin kreativitet innehade. Han arbetade bl.a som maskinist på Jungar mejeri. Skomakaruppdrag tog han emot fram till 1950-talet, då grävmaskinsarbeten upptog hans tid.

Bolaget ``Jeppo Gräv'' införskaffade en grävmaskin och Lennart samt Ensio Kula var de första maskinskötarna. Skomakarverktygen och -uppdragen övertogs av Uno Elenius. Lennart hade också en pärthyvel och en såg, som han körde runt med beroende på var uppdragen fanns. Nere vid älven hade han en smedja, som speciellt sommartid besöktes då verktyg behövde repareras.

Lennart byggde huset 1951. Tidigare bodde han med sin familj i hus \jhbold{nr 103}, nedan. I samma gård bodde brodern Evert och hans familj.

Lennart \textdied 29.06.1980  --  Edit \textdied 03.03.1982.



\jhhouse{Norrgård}{9:49}{Mietala}{14}{103}

\jhoccupant{Österbottens amb.}{Hemslöjdsskola}{1951-1952}
Österbottens ambulerande hemslöjdsskola var inhyst i huset efter att familjerna Norrgård byggt varsitt eget hus. Clarence Back från Kimo var lärare. Skolan hade 9 elever från orten: Valfrid Häggstrand, Ruben Eklöv, Uno Gunell, Evert Norrgård, Yngve Forsgård, Kurt Forsgård, Manne Strand, Gunnar Forsbacka och Jürgen Nylind. Man slöjdade köksinredningar, allmogebord, bänkar, sängar mm. Huset revs år 1958 av ägarna Lennart och Evert Norrgård.


\jhoccupant{Norrgård}{Gustaf \& Anna}{1898-1923}
Bonden Gustaf Johan Jakobsson Norrgård, \textborn 01.12.1878 på Mietala, gift med Anna Simonsdotter Lavast, \textborn 11.04.1879.
\begin{jhchildren}
  \item \jhperson{Anna Irene}{12.03.1903}{}, till Åland som tjänarinna, gift där
  \item \jhperson{Gustav \jhbold{Evert}}{01.02.1905}{}
  \item \jhperson{Valfrid}{20.04.1907}{03.05.1907}
  \item \jhperson{\jhbold{Lennart}}{18.06.1908}{}
  \item \jhperson{Sigurd Johannes}{25.12.1910}{}, (Romar 36 )
  \item \jhperson{Gerda Johanna}{19.08.1913}{}, till Åland 1936, gifte sig där
\end{jhchildren}
Genom afhandling 24 november 1898 blev Gustaf Johan Jakobsson Mietala ägare till 1/18 mantal Mietala hemman utav föräldrarna Johan Jakob Jakobsson Mietala och hustrun Johanna Andersdotter. Föräldrarna hade tre år tidigare överlämnat 1/18 åt brodern Anders.
\jhhousepic{Mietala103-Norrgard.jpg}{Gustaf och Anna Norrgård, hus 103}
Huset, som Gustaf fick överta, var byggt i Härmä, flyttades till Mietala nån gång i början på 1800-talet. Gustaf Norrgård deltog aktivt i den kommunala verksamheten, bl.a. var han åren 1910-1915 ordförande i kommunalstämman. Han var också invald i kommunalfullmäktige och hörde till olika kommunala nämnder. År 1923 överlät Gustaf hemmanet åt sönerna Evert och
Lennart. Dessa bodde med sina familjer i var sin ända av huset.Den 07.06.1915 dog Anna och Gustaf blev ensam med att ta hand om barnen. Den 02.11.1923 dog Gustaf och barnen måste klara sig själva. Deras farfar Johan Jakob Mietala hade dött 1918.


\jhoccupant{Johansson}{Johan Jakob \& Anna}{1853-1898}
Bondsonen Johan Jakob Johansson d.ä, \textborn 21.01.1831 på Mietala, vigd 06.11.1850 med bonddottern Anna Eriksdotter Fors, \textborn 06.11.1833. Anna \textdied 19.03.1858. Johan Jakob ingick nytt äktenskap 11.11.1860 med Johanna Andersdotter Kauhajärvi, \textborn 28.08.1837, \textdied 28.11.1898. Johans tredje hustru var Maria Johansdotter,  \textborn 06.03.1844.
Anna framfödde fyra barn av vilka två dog i tidig ålder.
\begin{jhchildren}
  \item \jhperson{Maria}{25.10.1851}{}, gift med Johan Lindqvist
  \item \jhperson{\jhbold{Johan Jakob}}{24.10.1853}{}, (Forsgård)
  \item \jhperson{Anna Sofia}{09.09.1855}{1861}
  \item \jhperson{Johanna}{11.01.1858}{1858}
\end{jhchildren}

Johanna födde åtta barn till familjen.
\begin{jhchildren}
  \item \jhperson{\jhbold{Anders} Wilhelm}{21.02.1861}{}, (Nord)
  \item \jhperson{Isak}{16.07.1862}{}
  \item \jhperson{Sofia}{23.08.1863}{}
  \item \jhperson{Simon}{09.07.1866}{}
  \item \jhperson{Johanna}{02.08.1870}{}
  \item \jhperson{Wilhelmina}{17.04.1873}{}
  \item \jhperson{\jhbold{Gustaf}}{01.12.1878}{}, (Norrgård)
  \item \jhperson{Otto}{01.11.1880}{20.12.1893}
\end{jhchildren}
Johan Jakob erhåller genom gåvobrev den 26 januari 1853 av sina föräldrar 1/9 dels mantal av Mietala hemman. I februari 1880 köpte makarna ett hemman av Simon Eriksson Mietala. Till detta köp hörde stugan samt hälften av alla andra hus. I juli 1884 brann stallet, ladan, en boda och ett loft ner. Om inte Otto von Essens brandspruta hämtats så snabbt, så hade en annan uthusbyggnad med all säkerhet blivit lågornas rov, enligt en notis i dagstidningen.

Jakob Mietala \textdied 23.05.1918, Maria flyttade året därpå till Jungar.


\jhoccupant{Henriksson}{Johan \& Maja Lisa}{1814-1853}
Bds Johan Henriksson, \textborn 31.07.1793 på Mietala, vigdes 07.04.1815 med Greta Eriksdotter, \textborn 22.09.1793. Tillsammans fick de ett barn; Henrik, \textborn 04.09.1815.
Greta \textdied 09.11.1818.

Johan ingick nytt äktenskap med Maja Lisa Simonsdotter Jungar, \textborn 19.05.1799.
\begin{jhchildren}
  \item \jhperson{Simon}{15.07.1820}{29.07.1820}
  \item \jhperson{Greta}{19.06.1826}{}
  \item \jhperson{Sanna Lisa}{26.10.1828}{14.04.1867}
  \item \jhperson{\jhbold{Johan Jakob}}{21.01.1831}{}
  \item \jhperson{Maja}{22.04.1833}{}
  \item \jhperson{Caisa}{21.07.1835}{}, till Alahärmä
  \item \jhperson{\jhbold{Sofia}}{25.12.1837}{}, (Mietala 116)
  \item \jhperson{Johanna}{01.11.1840}{}
\end{jhchildren}
År 1814 skrev Johan Jakobs mor och hennes andra man Carl Mietala över hemmanet åt sönerna Johan och Eric samt ansökte om klyvning av hemmanet. År 1836 köpte Johan 1/9 mantal av Mietala hemman av bonden Johan Thomasson Tolliko. I köpebrevet åläggs Johan att uppföra en stuga, 6 alnar i kvadrat, åt änkan Susanna Danielsdotter, Erik Henrikssons änka. Erik Henriksson, som var bror till Johan, dog 16.10.1833.


\jhoccupant{Henriksson}{Henrik \& Greta}{- 1814}
Henrik Henriksson, \textborn 11.03.1766 på Mietala, gift med Greta Mattsdr, \textborn 29.12.1766 i Kovjoki. Henrik dog 28.10.1798, Greta gifte om sig med Carl Eriksson Mietala, \textborn 21.10.1773.
\begin{jhchildren}
  \item \jhperson{Mathias}{30.01.1791}{04.02.1791}
  \item \jhperson{Maria}{15.07.1792}{}
  \item \jhperson{\jhbold{Johan}}{03.07.1793}{}
  \item \jhperson{Jakob}{06.09.1794}{19.09.1794}
  \item \jhperson{Henrik}{09.10.1795}{09.08.1796}
  \item \jhperson{Anders}{08.12.1796}{}
  \item \jhperson{\jhbold{Eric}}{17.01.1798}{}
\end{jhchildren}
År 1845 utgjordes Mietala av 2/3 mantal skatte- och 1/24 mantal kronojord. Samma år var Johan Henriksson samt medåboerna Johan Johansson Mietala och Anders Hansson gemensamma ägare till 1/3. År 1833 var Erik Henriksson och Johan Henriksson gemensamma ägare till 1/3 mantals skattehemman, år 1834 Johan Henriksson och Erik Henrikssons änka Susanna Danielsdotter.



\jhhouse{Nord}{9:88}{Mietala}{14}{4, 4a}

\jhoccupant{Nord}{Sven \& Britta}{1989 -}
Sven Nord, \textborn 21.04.1942 på Mietala, gifte sig 18.06.1977 med Britta Nymark, \textborn 06.10.1941 från Jakobstad. Makarna bor i Nykarleby och har inte bott i gården, som stått tom sedan Svens föräldrar dog. Sven skötte dock i många år lägenheten omfattande totalt 35 ha.

\jhhousepic{222-05800.jpg}{Äldsta gården i Jungar by och i hela Jeppo}
Enligt Runar Nyholm torde Nords gård vara den äldsta gården i Jeppo. ``Den har byggts på Jungar på 1600-talet, där den i ett par sekler utgjorde jungarsläktens stamgård – tolvmansgården. På 1800-talet byggdes en ny gård, och den gamla trotjänaren såldes åt Sven Nords förfäder, som har bebott den det senaste seklet. Efter vad en gammal man berättat, ville säljaren att gården skulle uppföras precis på samma sätt som den stått förut och med gaveln mot landsvägen. Rådet följdes till punkt och pricka, och därför kan man här se en exakt 1600-tals bondgård.'' En lång, smal och mörk farstu samt ett cirkelrunt litet fönster på gaveln mot öster är också tidstypiskt.


\jhoccupant{Nord}{Anders \& Elvira}{1947-1989}
Anders Vilhelm Nord, \textborn 30.11.1899 på Mietala, gifte sig 08.11.1936 med Anni Elvira Skog, \textborn 04.09.1912.
\begin{jhchildren}
  \item \jhperson{Viola}{21.04.1942}{15.08.2011}, gift med Boris Norrgård
  \item \jhperson{\jhbold{Sven}}{21.04.1942}{}
  \item \jhperson{Margareta}{04.01.1945}{}, gift med Jan-Erik Nybyggar
\end{jhchildren}
Anders byggde ekonomibyggnaden år 1954. På lägenheten funnits fähus, stall, kärrlada, sädesbod. Dessa var byggda i slutet på 1800 och revs 1955. Elvira var bland de första hårfrisörskorna i Jeppo.

Anders \textdied 09.05.1973  --  Elvira \textdied 18.02.1989


\jhoccupant{Jakobson}{Anders \& Anna}{1884-1947}
Anders Vilhelm Jakobson, \textborn 21.02.1861 i Jeppo, gift med Anna Sanna Jakobsdotter, \textborn 22.07.1862.
\begin{jhchildren}
  \item \jhperson{Johan Jakob}{15.10.1885}{}, till Amerika
  \item \jhperson{Anders Vilhelm}{21.05.1887}{02.04.1890}
  \item \jhperson{Otto}{03.09.1889}{06.01.1890}
  \item \jhperson{Hilda Johanna}{08.10.1893}{20.08.1894}
  \item \jhperson{Hilda Johanna}{01.12.1895}{17.04.1916}
  \item \jhperson{Anna Juliana}{23.12.1897}{27.07.1980}
  \item \jhperson{\jhbold{Anders} Vilhelm}{30.11.1899}{}
  \item \jhperson{Signe Maria}{04.11.1902}{13.11.1997}, g. Kytömäki
  \item \jhperson{Sanni Sofia}{01.12.1904}{05.01.1966}, g. Kronqvist
\end{jhchildren}
Anders och Anna får i sin ägo 1/18 mantal av Mietala hemman genom avhandling 16.03.1895 utav Anders föräldrar Johan Jakob Johansson och Johanna Andersdotter mot utlösen av andra syskon samt sytning åt föräldrar. Denna del hade föräldrarna handlat av Simon Eriksson Mietala (senare Sandbacka/Sandell) den 27 februari 1880. - Uppgifter om Johan Jakob Johansson d.ä. under Mietala 103.

Anna \textdied 23.03.1943  --  Anders \textdied 06.06.1947



\jhhouse{Dahl}{9:21}{Mietala}{14}{104}

\jhoccupant{Dahl}{Anna Lovisa}{1896-1944}
Anna Lovisa, \textborn 24.08.1854 i Esse, gift med Jakob Forsman, \textborn 29.11.1852 på Grötas. Jakob och Anna Lovisa bodde första åren på Ruotsala tillsammans med Jakob Johansson Forsman och hans familj, senare var de skrivna under lösa personer.
\begin{jhchildren}
  \item \jhperson{Anders Gustav}{06.01.1881}{}
  \item \jhperson{Johan Wilhelm}{03.03.1884}{}
  \item \jhperson{Mats Leander}{10.12.1886}{}
\end{jhchildren}
År 1896 flyttade Anna Lovisa in i torpstugan på Silta. Jakob torde tidigare ha emigrerat till USA, eftersom Anna 1892 		är antecknad som frånskild. Jakob gifte om sig i USA och fick där 5 barn. De tre sönerna från giftet med Anna Lovisa emigrerade senare till USA. Anna Lovisa får en son, Wiktor, år 1889, som dör 1890.

Anna Lovisa gifte om sig år 1893 med änklingen, f.d. skarpskytten Anders Peltomaa, senare Ekoluoma eller Dahl, \textborn 21.7.1837 i Kortesjärvi. Han hade 7 barn från sitt tidigare äktenskap. År 1892 efter hustruns död flyttade han till Jeppo med yngsta dottern Anna Adolfina \textborn 13.04.1883.

Anders Dahl \textdied 26.11.1908.

Efter att torparlagen gett möjlighet till inlösen av backstuguområden, ansökte Anna Lovisa om att få lösa in sitt område, som var 3 ¾ kappland. Området tillhörde bonden Jacob Forsgård. Det fanns inget skriftligt kontrakt, men i.o.m att Anna Lovisa årligen utfört 2 dagsverk ansågs det inlösningsbart. Dagsverksskyldigheten upphörde i samband med betalningen.

Adolfina gifte sig med Elias Kangas, \textborn 1881 i Kuortane och de flyttade en hel del mellan torpen på Jungar (1897 -09), Gunnar (1887-97) och Mietala. De bodde på Silta/Mietala tillsammans med Anna Lovisa åtminstone åren 1910-19. Elias emigrerade också till Amerika. Adolfina och Elias barn:
\begin{jhchildren}
  \item \jhperson{Anders Elias}{26.07.1903}{}
  \item \jhperson{\jhbold{Johannes Evald}}{31.05.1905 på Jungar}{}
  \item \jhperson{z}{z}{}
  \item \jhperson{z}{z}{}
\end{jhchildren}
Adolfinas övriga barnaskara gjorde familjen försvarligt stor.
\begin{jhchildren}
  \item \jhperson{Toivo}{1908}{}
  \item \jhperson{Helvi}{1911}{}
  \item \jhperson{Sulo}{1913}{}
  \item \jhperson{Jaakko}{1916}{}
  \item \jhperson{Antti}{1917}{}, (till Snappertuna 5.6.1941)
  \item \jhperson{Taisto}{1920}{}
\end{jhchildren}
Johannes Evald kom att bo tillsammans med Anna Lovisa under många år. Efter att ha gift sig med Hilda Stenbacka,  \textborn 03.10.1908, bodde han med sin nya fru ännu en tid kvar på Silta (1921-1930). De flyttade 1931 till Karleby. Evald var stationskarl och senare konduktör.

Anna Lovisa kallades ``Ståoanskon''. Som yngre arbetade hon som piga bl.a på Böös. Äldre personer kommer ihåg att hon tillverkade kvastar, som hon sålde. Man kommer ihåg henne komma dragande med kvastarna i en barnvagn. Bonden Fagerholm höll henne med en liten äng bredvid Björkvalls ria. Där hade hon fåren att beta.  Senare byttes denna äng ut till ett likadant område på Bomossan.

Adolfina \textdied 06.10.1927  --  Anna Lovisa Dahl \textdied 19.03.1944


\jhoccupant{Björn}{Tobias \& Sanna}{1875-1898}
Backstugukarlen Tobias Björn, \textborn 09.03.1856 i Sysmä, gift med Sanna Isaksdotter Löfqvist, \textborn 17.07.1844 i Alahärmä.
Från första giftermålet hade Sanna tre barn.
\begin{jhchildren}
  \item \jhperson{Susanna Karl Gustafsdr}{01.09.1870}{}, till Helsingfors 1890
  \item \jhperson{Anna Lovisa Löfqvist}{05.10.1874}{}, till Karstula 1892
  \item \jhperson{Katrina Löfqvist}{30.07.1877}{}, till Amerika 1896
\end{jhchildren}

Barn med Tobias: Amanda, \textborn 25.08.1886
Av kyrkböckerna får man den uppfattningen att Tobias Björn arbetat på annan ort. År 1888 flyttar han till Savolax, men hustrun och barnen besökte nattvard ännu 1897 och 1898. Vid denna tidpunkt har Sanna fått betyg till Amerika.


\jhoccupant{Heikfolk}{Gustaf \& Caisa}{1850-1880}
Gustaf Isaksson Heikfolk, \textborn 08.11.1821 i Ytterjeppo, gift med Caisa Greta Eriksdotter, \textborn 22.12.1817 i Ytterjeppo.
\begin{jhchildren}
  \item \jhperson{Susanna}{11.11.1843 i Ytterjeppo}{}
  \item \jhperson{Catharina}{19.10.1845 i Kauhava}{}
  \item \jhperson{Isak}{12.08.1847 i Ytterjeppo}{}
  \item \jhperson{Maria}{30.08.1849  ''}{}
  \item \jhperson{Erik}{24.09.1851 på Mietala}{}
  \item \jhperson{Johan Jakob}{07.08.1854  ''}{}
  \item \jhperson{Lovisa}{03.09.1857  ''}{}
  \item \jhperson{Anna}{30.12.1859  ''}{}
  \item \jhperson{Gustaf}{22.10.1862  ''}{}
\end{jhchildren}
I köpekontrakt mellan Johan Jakob Johansson d.ä och sonen Johan Jakob (senare Forsgård) samt hans hustru Anna Isaksdotter år 1879 ``tryggas torparen Gustaf Isakssons torpkontrakt''.
