
KARTA nr 7 placeras här ------>


\jhhouse{Kotimökki}{3:38}{Fors}{7}{129}

\jhoccupant{Asplund}{Benny \& Åsa}{2016 -}
Benny Asplund, \textborn 21.02.1972, gift med Åsa Frilund, har fr.o.m 2016 övertagit fastigheten, men familjen bor i Lippjärvi, Nykarleby.


\jhoccupant{Aslund}{H \& G dödsbo}{2014-2016}
Som sterbhus har huset stått obebott.


\jhoccupant{Asplund}{Helge \& Gurli}{1957-2014}
Helge Johannes Asplund, \textborn 29.01.1933 i Pensala, gifte sig med Gurli Anita Juslin, \textborn 15.01.1935 i Jeppo. De köpte huset av Erik och Hjördis Haga år 1957.

Helge började som 13 åring arbeta på skinnfabriken vid Kiitola. När den lades ner fungerade han till en början som diversearbetare och arbetade med manuell skogsdikning innan han fick anställning inom det på orten nystartade företaget Mirka, ett företag han var trogen fram till sin pensionering. Gurli har haft hand om skötseln av hemmet och familjen med de 3 sönerna. Därtill har hon arbetat med städning vid både Posten och Mirka.

\jhhousepic{Asplund.jpeg}{}

\begin{jhchildren}
  \item \jhperson{Bengt}{16.03.1954}{}, har länge jobbat som montör på Jeppo Kraft, bor i Nykarleby
  \item \jhperson{Boris}{30.07.1959}{}, bor i Sverige
  \item \jhperson{Benny}{21.02.1972}{}, se ovan
\end{jhchildren}

Helge \textdied 17.03.2014  --  Gurli \textdied 10.07.2016


\jhoccupant{Haga}{Erik \& Hjördis}{1951-1957}
Erik Alfred Haga, \textborn 14.05.1923 på Skog, gifte sig 10.04.1944 med Hjördis Margareta Backlund, \textborn 13.11.1922 i Kimo. Under en permission den 1 maj 1943 träffade han sin blivande hustru Hjördis på ``danslavan'' i Gränden. Hjördis hade då sedan år 1942 varit anställd som kontorsbiträde på Handelslaget.

Efter att ha legat på militärsjukhuset i Vasa, åkte Erik i juni 1944 till Stockholm för att delta i en sex månaders kurs för mekaniker, en kurs som vände sig till krigsskadade finländare. I huset bodde då en familj som evakuerats från Kemijärvi. Den bodde kvar till sommaren 1945. Därefter flyttade familjen Haga tillbaka till huset.

Efter kriget köpte Erik hemmansdelar av sina syskon. Tillsammans med Hjördis drev han ett litet jordbruk och under en period satsade de på  hönseri. Samtidigt arbetade Erik på Kiitola pälsberederi. År 1952 köpte han sin första taxibil av märket Austin. Hjördis arbetade under åren 1955-1957 på Andelsbanken. 1957 såldes huset på Holmen. Erik och Hjördis köpte Åkermarks fastighet på stationsvägen vilken kom att inrymma Kemikalia, affär och medecinskåp, se nr 77.
\begin{jhchildren}
  \item \jhperson{Eivor Inga-Lisa}{17.10.1944}{}
  \item \jhperson{Monica Elisabeth}{28.12.1948}{}
  \item \jhperson{Ingmar Erik Johan}{08.08.1951}{}
  \item \jhperson{z}{z}{}
\end{jhchildren}


\jhoccupant{Haga}{Erik, Margit \& Evald}{1943-1951}
Erik Alfred Haga, \textborn 14.05.1923,  Margit Susanna Haga, \textborn 01.12.1925 och brodern Johannes Evald Haga, \textborn 04.12.1928, övertog fastigheten efter faderns död 18.12.1943. Erik och Hjördis köpte sedan fastigheten 01.12.1951.


\jhoccupant{Haga}{Johan}{1939-1943}
Marg
Johan Mariasson Haga, \textborn 05.04.1883 i Ylistaro, men bosatt på Skog, köpte den 18.06.1939 en tomt på Holmen, tillhörande Fors hemman nr 3 av Torsten Forss och köpesumman var 6000 mk. Huset timrades upp och kom att tjäna som bostad också åt de tre barnen Erik, Margit och Evald. Erik blev inkallad i kriget år 1942 och sårades i mars 1943. Margit och Evald studerade på annat ort.

Johan dog 18.12.1943 i huset på Holmen och de tre syskonen övertog fastigheten.



\jhhouse{Kotimökki 2}{3:124}{Fors}{7}{329}

\jhoccupant{Savolainen}{Juho & Elna}{1936- ..}
Stationskarlen Juho Konstantin Savolainen, \textborn 07.07.1897 i Töysä, gifte sig 20.05.1922 med Elma Eliina Lempinen, \textborn 29.06.1898 i Ähtäri. De anlände från Alahärmä 04.12.1935 och bodde en tid på stationsområdet innan Juho inköpte ett hus på Holmen som byggts av Erik Nyman, svärson till Mårten och Brita Kajsa Theel och där dessa bott på sin ålderdom. Erik Nyman sålde sin lägenhet 1929 till Henrik Johansson Jungell, som överlät denna till sonen Birger Jungell.

Det hus som Mårten och Brita Kajsa bott i, köptes småningom av Juho Savolainen och timrades upp på den från Torsten Forss 16.11.1936 inköpta tomten bredvid bönehuset. Här bodde Juho till sin död. Hans hustru avled 1955 och deras dotter Kaija, \textborn 14.05.1927 växte upp här. Sedan Kaija gift sig 1949 flyttade hon till Puolanka, men återvände som änka därifrån 17.12.1952 med sin dotter Tuula, \textborn 15.02.1950. De bodde då med Juho och Elma tills de flyttade till Polvijärvi ett år senare.

Juho \textdied 08.10.1972  --  Elma \textdied 02.12.1955



\jhhouse{ Åbrant}{3:45}{Fors}{7}{130}
