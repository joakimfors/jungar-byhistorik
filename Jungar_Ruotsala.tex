\jhchapter{Jungar \& Ruotsala, hemman Nr 7 och 8}

\jhbold{Jungar, hemman  Nr 7}

Vi har redan i början av denna historik behandlat Jungar-namnet och dess osäkra ursprung.
I likhet med hemmanen Silvast och Fors går också hemmanen Jungar och Ruotsala om varandra i geografiskt hänseende. Jungar börjar med en något hackig gräns strax efter  Bösas bäck och når fram till vägen som leder ner mot ån efter Tom Jungerstams driftcentrum. Där tar Ruotsala hemman vid och fortsätter förbi Benny Gunells gård, varefter Jungar hemman på nytt kommer in i bilden. Dan Backs bostad öster om landsvägen är den sista på Jungar hemman, men Kjell Liljeqvists gård strax före det, är en  insprängd del av Ruotsala. På västra sidan av vägen börjar Mietala vid Kjell Forsgårds. Inte lätt! De bifogade kartutsnitten följer därför inte alltid hemmansgränserna.

Jungar var alltså vid början av den nya tiden ett vedertaget namn på ett hemman om 1 mtl i Jäpu. Det är rimligt att anta att det funnits redan vid den skattläggning, som Gustav Vasa föranstaltade efter sitt maktövertagande 1521. Var fanns då Jungar hemman beläget? Idag associeras namnet Jungar i första hand till den gårdsgrupp som fortsättningsvis bär det namnet strax söder om Bösas bäck. Det finns ändå skäl att tro att bostället , d.v.s. platsen med bostad, fähus, lider, bodar och visthus funnits ett stycke längre söderut. Detta på basen av gamla kartor.

Hemmanet klövs, sannolikt 1644, mellan bröderna Jakob och Mårten Mattsson. Den nya hemmansdelen omfattande ½ mtl övergick till Mårten. Denna del fick nu ett annat namn, nämligen Rautzinkåski (Ruotsinkoski), senare Ruotsala. Det är troligt att den geografiska fördelningen har sin ursprung i denna klyvning.

Enligt ``Släkt och bygd'' och Pär-Erik Levlin har hemmanet ägts sedan 1582 av:
Hans Larsson  1582-1607 (hade 6 kor), Markus Hansson 1608-1627, Matts Hansson 1627-1644 varvid hemmanet klövs, Jakob Mattsson (son) 1644-1651, Olof Jakobsson, hu. Lisa Eriksdr. 1654-1673, Hans Jakobsson (bror), hu. Lisa Johansdr. 1675-1681, Matts Hansson, hu. Lisa Johansdr. Medåbo brorson Simon Olofsson, hu. Brita Henriksdr. 1683-1697, svägerskan Brita 1698, Gustav Matsson, hu. Brita 1699-1710, Jakob Matsson, hu. Maria 1711-1719, änkan Maria 1723-1725, Gustav Jakobsson (son) 1725-1760, Daniel Gustafsson 1760-1790, Daniel Danielsson 1790-1828, Johan Danielsson 1828-1874, Isak Johansson 1874-1896. Hemmanet har efter tid styckats i flera delar, vilket framgår ur gårdsbeskrivningen.

Bosättningen på Jungar hemman har länge expanderat i jämn takt och var år 1900 störst i Jeppo med 141 pers. Beaktar man att samtidigt på grannhemmanen Ruotsala och Mietala fanns 119 resp 113 personer, var koncentrationen av människor på en sträcka om ca 1,5 km ansenlig. Detta år fanns det bönder 59 st, torpare 48, tjänstefolk 20 och hantverkare 14 på Jungar. Jungar hemman deltog i rote tillsammans med Gunnar, Krögarn, Eckolla (Ekola) och Kojonen hemman och hade ansvar för soldattorp 141. Det benämndes över tid med namn som Snarubacka,Nårrbacka och Jungar. Under gård nr 22 (Senny Nylund) nämns om att den platsen skulle ha hyst ett litet soldattorp, men uppgiften är osäker (se Gunnar hemman).

Jungar-namnet är också kopplat till ett gästgiveri/krog. Sedan 1677 låg Jungar krog på älvens östra sida vid Holmens södra spets. När  drygt 80 år senare handlanden Johan Blad från Vasa i samband med ett planerat upprättande av ett pappersbruk i Överjeppo, ville upprätta en krog mitt emot Jungar krog,  på älvens västra sida, var stridigheter att vänta. Mera därom under Jungarå hemman.

Från Jungar har utgått många dugliga människor genom tiderna. Ättling till en av dessa backstugusittare, som på 1700/1800-talet flyttat bort, är Mikko Juva och var Finlands ärkebiskop 1978-1982. Han dog 2004. På Jungar har i övrigt verkat många duktiga personer som satt sin prägel på Jeppo-bygden. Idag finns kvar 3 aktiva brukningsenheter på detta hemman, som fortfarande har kvar sin agrara karaktär.


\jhbold{Ruotsala, hemman  Nr 8}

I samband med storskiftet i slutet av 1700-talet noteras hemmanet under nummer 30 och har då benämningen Ruotsinkoski. På lantmäterikartor från 1870-talet finns hemmanet nämnt som både 30 I \= Ruotsinkoski och som 30 U \= Ruotsala. Deras gemensamma nya nummer är ändå Nr 8, Ruotsala. Namnet Ruotsala har antagits betyda ``stället där svenskar bor'' och kommit i användning efter Stora ofreden i början av 1700-talet, när inflyttande finnar övertog öde hemman. Men numera anses det vara äldre än så. Det går tillbaka åtminstone till 1600-talet.

Hemmanet har omfattat ½ mtl och är därmed ett av de mindre hemmanen. År 1644 delades nämligen Jungar hemman, som tidigare utgjort 1 helt mantal, och därigenom uppstod Ruotsala hemman. Sedan 1644 har bl.a. följande ägt detta hemman:
Mårten Mattsson 1646-1678, sonen Markus Mårtensson 1681-1697 och hans änka Malin Hansdr. 1698-1699, Mårten Mårtensson  1700, Mårten Markusson med hu. Margareta 1701 -. Medåbo är då brodern Matts Markusson med hu. Margareta Karlsdr. År 1783 ägdes hemmanet av Johan Jakobsson 5/24 mtl och Erik och Johan Eriksson 5/24 mtl vardera. År 1850 bodde här 58 personer.

Som framgår av Jungar hemmans beskrivning går hemmanen om varandra och har säkert sin bakgrund i det faktum att hemmanet delades mellan bröderna Jakob och Mårten Mattsson 1644. Sannolikt försökte de finna en överenskommelse om en någorlunda rättvis jordfördelning vad gällde bördighet och skick och därför blev resultatet detta. Vid all jordrevning d.v.s uppmätning och kartläggning av mark, var rättviseaspekten viktig när det gällde skatteläggningen.

Ruotsala deltog i rote tillsammans med hemmanen Storsilfwast och Måtare. Även om själva torpet fanns på Storsilfwast, fanns torpets ängar med säkerhet utspridda på de andra hemmanens marker och de uppges 1791 vara i gott stånd och avkasta 12 skrindor hö.

En alldeles speciell plats på Ruotsala hemmans marker är ``Svartbackan''. Nu finns endast en bostad på denna plats, men under 1800-talet och också in på 1900-talet fanns här en riklig bosättning för torpare och backstugusittare, d.v.s. personer utan eget jordinnehav. Det sägs att här har mer än 20 bostäder fått plats med rika barnfamiljer. Vilka alla som här har bott kan vi inte redogöra för, men en del av dem dyker upp i gårdsbeskrivningarna. Många har i tiden emigrerat och husen lämnats öde för att antingen flyttas närmare ån eller rivas. I mantalsförteckningarna över ``lösa personer'' från slutet av 1800-talet finns antecknat att också Ryssland var en populär plats att vistas på och där tjäna sitt uppehälle, precis som för andra byars människor.

Jungar och Ruotsala hemman omfattas av vidstående karta nr \jhbold{13}.


KARTA nr 13 hit --->


\jhsubsection{Lägenheter på Jungar}


\jhhouse{Torrlandet}{7:69}{Jungar}{13}{3}

\jhoccupant{Strengell}{Markus}{2009 -}
Markus Erik Gustav Strengell, \textborn 05.09.1982, övertog sina föräldrars bostad 2009 med en tomt på 6000 m2. Han lever ogift och är anställd som lantbruksarbetare på sin bror Glenns lägenhet på Jungar (se nr 5).

\jhhousepic{190-05908.jpg}{Strengell E., A-M och Markus}

\jhoccupant{Strengell}{Erik \& Ann-Mari}{1978 -}
Erik Ingmar Strengell, \textborn 12.10.1935, gifte sig 22.12.1963 med Ann-Mari Ekman, \textborn 22.05.1943 i Oravais. Bostaden uppfördes 1978 en liten bit från landsvägen och driftcentrum. Den gamla mangårdsbyggnaden lämnades kvar bredvid landsvägen och grundrenoverades till bostad för följande generation.



\jhhouse{Norråker}{7:60}{Jungar}{13}{5}

\jhoccupant{Strengell}{Glenn \& Christina}{1993 -}
Glenn Mikael Strengell, \textborn 17.12.1964 på Jungar gifte sig 23.06.1990 med Christina West, \textborn 28.03.1967 i  Oravais. Glenn är \jhbold{nr 21 i obrutet släktled} från 1593 som innehaft gården!

\jhhousepic{193-05763.jpg}{Strengell mangårdsbyggnad}
År 1993 övertog de Glenns hemgård, som genom åren avsevärt förstorats och idag utgörs av ca 190 ha skogsmark, 96 ha egen odlad mark och 22,5 ha arrende. Gården har specialiserat sig på potatis och sköter utöver odling också sortering, förpackning och marknadsföring av produkten. Spannmålsodlingen är också omfattande. Gården har därtill utvecklat inkvarteringstjänster i 2 separata hus. Familjen bor i hemmanets gamla mangårdsbyggnad uppförd 1896 och numera pietetsfullt renoverad. I detta hus har släkten bott i mera än 120 år och härifrån flyttade Glenns föräldrar 1978 till sitt nybyggda hus (se nr 3).

Christina är utbildad sjukskötare, Glenn lantbruksutbildad vid Korsholms skolor. Makarna är engagerade i ett flertal föreningar och andra uppdrag. Glenn bl.a. som mångårig ordförande i Jeppo lokalavdelning av ÖSP. Han har suttit i Nykarleby stadsfullmäktige och i Jeppo Foods styrelse. Han är också ordf. för Jeppo Skogsandelslag och Jeppo SFP-avd.

Christina har medverkat i många Jepporevyer och var med vid bildandet av ``Jeposmettona'', en förening för yngre kvinnor, liksom styrelsemedlem vid nystarten av Jeppo Byaråd 2003-2007, nu Jeppo Byaförening. Hon har fotvårdsmottagning, sysslar med turism och hyr ut bl.a. ``Lovisastugan'' (se Böös 55) för övernattningar i Jeppo.
\begin{jhchildren}
  \item \jhperson{Erika Maria}{23.03.1991}{}
  \item \jhperson{Jakob Mikael}{04.07.1993}{}
  \item \jhperson{William Carl Gustaf}{18.03.1997}{}
  \item \jhperson{Viktor Glenn Christian}{27.09.2008}{}
\end{jhchildren}


\jhoccupant{Strengell}{Erik \& Ann-Mari}{1960-1993}
Erik Ingmar Strengell, \textborn 12.10.1935 på Jungar, gifte sig 22.12.1963 med Ann-Mari Ekman, \textborn 22.05.1943 i Oravais. Makarna övertog 1960 Eriks hemgård av hans föräldrar. Odlingen var till en början traditionell med mjölkkor och ungnöt. Småningom i takt med att ny mekanisering visade nya möjligheter, ökade också intresset för potatisodling och efter ingången av 1970-talet fick korna stryka på foten och gården blev kreaturslös. Satsningen på potatis blev nu huvudfåran och odlingen utökades successivt.

Erik har varit engagerad i samhället. Han har bl.a. suttit med i Jeppo Kommunalfullmäktige, ÖSP:s styrelse, Ungdomsföreningen och älgjaktlaget. Som änkling bor han nu på Florahemmet i Nyk:by.
\begin{jhchildren}
  \item \jhperson{\jhbold{Glenn} Mikael}{17.12.1964}{}
  \item \jhperson{Sören Anders}{01.11.1966}{06.11.1991}
  \item \jhperson{Camilla Ann-Christine}{11.12.1971}{}
  \item \jhperson{Markus Erik Gustav}{05.02.1982}{}
\end{jhchildren}
Ann-Mari \textdied 30.04.2015


\jhoccupant{Strengell}{Johannes \& Elna}{1928-1960}
Johannes Sigfrid Strengell, \textborn 28.09.1905, gifte sig 17.06.1928 med Elna Johanna Sandell, \textborn 29.11.1906 på Måtar. Efter att bröderna, Alfred, Joel, och Simon avstått från att ta hand om fädernegården var Johannes den sista av familjens söner kvar.  Han upplevde det som ett kall och även om ekonomin var körd i botten övertog han gården med allt det slit som avsaknaden av motordrivna maskiner innebar.

Genast efter bröllopet flyttade Elna  in i ``den på kvinnfolk tomma gården'' och tog itu med arbetet, som naturligtvis aldrig tog slut. Johannes byggde en smedja på Torrlandet bredvid Böösas bäcken där han ofta tillbringade långa stunder med mekaniska göromål. Han blev en skicklig smed som inte hade någon tilltro till  den moderna svetsningen. Den var inte hans likör. Förvällning var det som gällde, d.v.s. de järnstycken som skulle sammanfogas hettades i fogområdet upp till omedelbar närhet av smältpunkten och smiddes snabbt ihop med hammare på städ. ``Nu håller det'', sa Johannes. Smedjan revs i medlet av 1970-talet. Under Johannes och Elnas tid förkovrades hemmanet, vilket fortsatt.
\begin{jhchildren}
  \item \jhperson{Ragni Alice}{13.11.1928}{}
  \item \jhperson{Wolmar Gustav}{05.12.1929}{}
  \item \jhperson{\jhbold{Erik} Ingmar}{12.10.1935}{}
  \item \jhperson{Gunda Gertrud Monika}{22.09.1937}{}
\end{jhchildren}
Elna \textdied 04.08.1978  --  Johannes \textdied 02.10 1989


\jhbold{1907-1928}
Gustav Strengells arvingar utgör under denna period det 18:e släktledet och efter att hans änka Maria gift om sig och 1914 flyttat till Rundt (se nedan) sköttes hemmanet av dottern Helmi och sonen Johannes, som inte flyttade med. Styvfar Johannes Granlund hjälpte också till, men hemmanet förföll till en del under denna tid.


\jhoccupant{Strengell}{Gustav \& Maria}{1899-1907}
Gustav Andersson Jungar, \textborn 08.01.1871 gifte sig med Maria Bro, \textborn 28.02.1872 i Nykarleby. Giftermålet skedde 14.08.1892 och Maria flyttade nu till Jeppo. De bosatte sig som nygifta i den 1896 nybyggda fädernegårdens ena ända. 1899 övertog Gustav ½ av hemmanet. Brodern Anders övertog den andra delen. Under denna tid ändrades efternamnet från Jungar till \jhbold{Strengell}. Äktenskapet blev kort p.g.a. Gustavs för tidiga död 23.07.1907 p.g.a. minsjuka, men 6 barn hann födas.
\begin{jhchildren}
  \item \jhperson{Johan August Eliel}{1893}{1902 i hjärtfel}
  \item \jhperson{Gustav Alfred}{06.07.1895}{}
  \item \jhperson{Anders Joel}{03.02.1898}{09.07.1919 (i spanska  sjukan)}
  \item \jhperson{Helmi Elisabet}{28.05.1900}{}
  \item \jhperson{Simon Severin}{27.06.1902}{}
  \item \jhperson{\jhbold{Johannes Sigfrid}}{28.09.1905}{}
\end{jhchildren}
Maria gifte om sig 11.07.1909 med Johannes Rundt Granlund, \textborn 24.09.1862 på Rundt. 1914 flyttade familjen till Rundt, men på Jungar föddes ytterligare 2 barn innan det 3:e föddes på Rundt.
\begin{jhchildren}
  \item \jhperson{Helga}{12.11.1910}{}
  \item \jhperson{Lea}{26.12.1911}{}
  \item \jhperson{Edit}{30.11.1916}{}
\end{jhchildren}


De 13 första släktleden är gemensamma för flera av de nuvarande hemmanen på Jungar. 1790 delas hemmanet mellan bröderna Simon Danielsson och Daniel Danielsson. Simon Danielssons hemmansdel delas på nytt 1899 mellan bröderna Anders och Gustav Andersson och det är Gustavs hemmansinnehav vi följt ovan. Hans föregångare ses i förteckningen nedan.
\begin{center}
  \begin{tabular}{l l l l}
    \hline
    Markus Olofsson & 1593-1604 & Jakob Mattsson(bror) & 1708-1722 \\
    Hans Markusson & 1604-1621 & Gustav Jakobsson & 1722-1760 \\
    Markus Hansson & 1621-1627 & Daniel Gustavsson & 1760-1790 \\
    Matts Hansson(bror) & 1627-1635 & Simon Danielsson & 1790-1826 \\
    Jakob Mattsson & 1635-1660 & Anders Simonsson & 1826-1845 \\
    Olof Jakobsson & 1660-1668 & Simon Andersson & 1845-1866 \\
    Hans Jakobsson(bror) & 1668-1681 & Anders Simonsson & 1866-1899 \\
    Matts Hansson & 1681-1697 & \jhbold{Gustav Andersson} & 1899-1907 \\
    Gustav Mattsson & 1697-1708 &  &  \\
    \hline
  \end{tabular}
\end{center}
Hemmanets öden har kunnat spåras ännu längre tillbaka i tiden, ända till 1382, men uppgifter om ägarna saknas.



\jhhouse{Jungar}{7:52}{Jungar}{13}{7}

\jhoccupant{Strengell}{Ruben}{1975 -}
Anders Ruben Strengell, \textborn 04.10.1942 övertog hemmanet 1975 som den 19:e odlaren i släktledet. Under sin aktiva tid fortsatte Ruben tillsammans med sin syster Elvi den mjölkproduktiom som hade funnits på gården. Den avvecklades år 2007 och odlingsjorden utarrenderades till systersonen Lasse Mäenpää, som är jordbrukare och pälsfarmare i Kauhava.

\jhhousepic{195-05765.jpg}{Ruben Strengell}
Ruben är ogift medan Elvis tidigare äktenskap är upplöst. Denna del av Jungar hemman, som har samma ägohistoria som flera andra hemman i byn, utgörs av 18,25 ha åker och 32,5 ha skog. Den nuvarande bostadsbyggnaden uppfördes 1977 av Ruben. Den gamla karaktärsbyggnaden uppförd i medlet av 1800-talet revs år 1980.


Jungar hemman   Karta 13     nr \jhbold{107}

\jhoccupant{Strengell}{Georg \& Dagny}{1932-1975}
Georg Valdemar Strengell, \textborn 28.08.1908 på Jungar, gifte sig 25.06.1936 med Dagny Johanna Berg, \textborn 01.12.1910 på Kojonen i Lassila. Redan som 8 åring överfördes hemmanet på hans axlar p.g.a. faderns ohälsa. 1929 reste han till Canada och återvände 1932 varefter han övertog hemmanets skötsel och gifte sig. Ladugården av kilad sten har renoverats och ekonomiebyggnaden uppfördes 1942 mitt under kriget. Mjölkproduktionen blev familjens huvudsakliga inkomstkälla. Georg har också suttit med i hälsovårdsnämnden.
\begin{jhchildren}
  \item \jhperson{Olof Georg}{21.01.1938}{}
  \item \jhperson{Elvi Dagny Anita}{25.08.1939}{}
  \item \jhperson{Anders \jhbold {Ruben}}{04.10.1942}{}
  \item \jhperson{Doris Carita}{28.04.1950}{}
\end{jhchildren}
Georg \textdied 07.04.1987  --  Dagny \textdied 17.03.1993


\jhoccupant{Jungar}{Anders \& Maria}{1899-1916}
Anders Andersson Jungar, senare Strengell, \textborn 19.09.1873, gifte sig 20.10.1907 med Maria Sofia Pettersdr. Pensar \textborn 14.06.1877 i Pensala. Anders löste den 14.02.1902 biljett från Hangö till Kapstaden och reste därmed ner till gruvorna i Sydafrika. Han återvände efter 2 år, men reste på nytt 03.12.1904 och drabbades där som så många andra av `minsjukan''. Han hade övertagit ½  hemmanet 1899. 1907 gifte han sig och redan 1917 dog han av sviterna från arbetet i gruvorna och hemmanet kvarstod som sterbhus tills sonen Georg år 1932 övertog skötseln.
\begin{jhchildren}
  \item \jhperson{\jhbold{Georg Valdemar}}{28.08.1908}{}
  \item \jhperson{Oskar Evald}{21.11.1910}{}
  \item \jhperson{Simon Edvin}{27.03.1913}{}
  \item \jhperson{Gustaf Runar}{04.07.1915}{}
  \item \jhperson{Astrid Gustava}{10.09.1920}{}
\end{jhchildren}
De 13 första släktleden är gemensamma för flera av de nuvarande hemmanen på Jungar. 1790 delas hemmanet mellan bröderna Simon Danielsson och Daniel Danielsson. Simon Danielssons hemmansdel delas på nytt 1899 mellan bröderna Anders och Gustav Andersson och det är Anders hemmansinnehav vi följt ovan.
\begin{center}
  \begin{tabular}{l l l l}
    \hline
    Markus Olofsson & 1593-1604 & Jakob Mattsson(bror) & 1708-1722 \\
    Hans Markusson & 1604-1621 & Gustav Jakobsson & 1722-1760 \\
    Markus Hansson & 1621-1627 & Daniel Gustavsson & 1760-1790 \\
    Matts Hansson(bror) & 1627-1635 & Simon Danielsson & 1790-1826 \\
    Jakob Mattsson & 1635-1660 & Anders Simonsson & 1826-1845 \\
    Olof Jakobsson & 1660-1668 & Simon Andersson*) & 1845-1866 \\
    Hans Jakobsson(bror) & 1668-1681 & Anders Simonsson**) & 1866-1899 \\
    Matts Hansson & 1681-1697 & \jhbold{Anders Andersson} & 1899-1917 \\
    Gustav Mattsson & 1697-1708 &  &  \\
    \hline
  \end{tabular}
\end{center}
*) \textborn 19.11.1812 -- \textdied 16.08.1868; g m Cajsa Anderdr. Bärs, \textborn 26.08.1814 -- \textdied 1870-talet

**) \textborn 27.07.1836 -- \textdied 28.01.1909; g m Sanna-Lisa Johansdr Mietala, \textborn 10.04.1835 -- \textdied 23.02.1906

Hemmanets öden har kunnat spåras ännu längre tillbaka i tiden, ända till 1382, men då med andra släkten som ägare.



\jhhouse{Älvdal}{7:42}{Jungar}{13}{8}

\jhoccupant{Jungar}{Bengt \& Monica}{1970-talet}
Bengt Gustav Jungar, \textborn 16.04.1953, gifte sig 2013 med Monica Blomqvist, \textborn 06.07.1961 i Nykarleby. Redan som 8 åring blev Bengt faderlös sedan fadern plötsligt avlidit i hjärtattack under utearbetet 1961. Tillsammans med sin mor Linnea skötte han i unga år om jordbruket tills han övertog hemmanet på 1970-talet

\jhhousepic{197-05764.jpg}{Jungar B och M}

Bengt har varit yrkeschaufför på tunga fordon med start som chaufför vid Forss transportfirma i Munsala, därefter hos Holm´s transportfirma vid Jutas och från början av 1980-talet hos Nyko Frys.

Sedan Monica flyttade till Jungar i medlet av 1980-talet har hon arbetat inom åldringsvården både på Östervall och Hagalund åldringshem i Nykarleby. Därefter har hon haft anställning på Mirka för att sedan i 7 års tid arbeta hos Nykarleby Lastbilscentral. För tillfället arbetar hon på distans för speditionsfirman Mixell Logistics i Närpes som transportsekreterare.
\begin{jhchildren}
  \item \jhperson{Louise Blomqvist}{02.12.1979}{}, g Holmäng, jobb i P:öre församling
  \item \jhperson{Johnny Jungar}{04.07.1987}{}, arbetar på Mirka
\end{jhchildren}


\jhoccupant{Jungar}{Runar \& Linnea}{1939-1961}
Gustav Runar Jungar, \textborn 10.09.1912 på Jungar gifte sig 22.10.1950 med Anna Linnea Häggstrand, \textborn 07.01.1912 på Mietala. Hemmanet utgjorde ½ av föräldrarna Daniel och Anna-Sofias hemman som han delat med brodern Sigurd den 23.02.1939, omfattade 30 ha skog och 13 ha jord. Jordbruket sköttes traditionellt men Linnea hade redan tidigare fungerat som kontrollassistent och fortsatt med detta efter makens död.

Allt ställdes på ända när Runar avled den 3 november 1961. Linnea stod nu ensam med 2 barn. Hemmanet hölls dock kvar i familjens ägo, men utarrenderades. Efter att Bengt övertagit lägenheten flyttade Linnea 1987 till Nykarleby.
\begin{jhchildren}
  \item \jhperson{\jhbold{Bengt} Gustav}{16.04.1953}{}
  \item \jhperson{Benita Ann-Katrin}{22.10.1954}{}
\end{jhchildren}
Runar \textdied 03.11.1961  --  Linnea \textdied 12.01.1996


\jhoccupant{Jungar}{Daniel \& Anna-Sofia}{1896-1939}
Daniel Jungar, \textborn 13.02.1875 på Jungar, gifte sig 1896 med Anna-Sofia Johansdr., \textborn 30.11.1876 i Markby. Han besökte Kronoby folkhögskola vintern 1893/94 vilket skulle påverka hans framtid. Två år senare, den 10.10.1896 delades föräldrarnas hemman i 2 delar varvid Daniel fick ½ och brodern Johan, \textborn 1867 fick den andra delen. Föräldrarna Isak Johansson, \textborn 1837 och Kajsa, \textborn 1833, stannade som inhyseshjon hos Daniel och Anna-Sofia. Till Amerika reste han 19.10.1901 sedan de 3 första barnen fötts. Hemkommen byggde han den mangårdsbyggnad som fortsättningsvis står kvar och bebos av hans sonson Bengt med familj.

År 1916 blev han polis samtidigt som han var engagerad i motståndet mot ryssarna, vars trupper var stationerade i landet. Tillsammans med bl.a. Carl Jonathan von Essen på Kiitola var han aktiv med att smuggla landsmän över till Sverige via Monäs för vidare transport till militärutbildning i Tyskland. Det var von Essen och kyrkoherden Westergren som föreslog honom till polis därför att det var viktigt att man som polis hade en person man kunde lita på. Ofta stod han vid stationen vid denna tid och kom med hemliga lösenord i kontakt med dem som skulle vidare. Det betydde att Daniel om dagarna tjänade ryssarna och om natten hjälpte jägarna. Att någon efter en tid läckte obekväma uppgifter till ryssarna väckte deras misstänksamhet och situationen blev farlig, men när väl frihetskriget startade 28 jan. 1918 var kurragömmaleken slut. Han deltog både i Gamlakarleby och Uleåborg vid avväpnadet av de ryska garnisonerna.

Daniel utvecklade senare ett facistiskt tankemönster och hade länge sympatier för nazi-Tyskland och han var också ``Lappo-man''. Hans åsikter delades inte nödvändigtvis av hans nära släktingar. Han var också en renlevnadsman och var 1905 grundande medlem av Jeppo nykterhetsförening, vars första ordförande han blev. Likaså deltog han aktivt vid genomförandet av telefonin i Jeppo  och var direktör för Jeppo Telefon Ab 1921-34.

Skyddskåren och dess verksamhet stod honom varmt om hjärtat och han ledde ett av distriktets övningar. Naturligt nog var han intresserad av vapen och jakt. Han var sträng, men inte elak, sa en av grannarna. Otvetydigt var han en respektfull gestalt och många ungdomar var rädda för honom fastän han flera gånger yttrade: ``An ska it vaar träätand mi hundan å smoåpåjkan''.

Han deltog gärna i diskussioner och var envis i sina åsikter utan att fördenskull dominera. Han ansågs vara en sympatisk sällskapsmänniska och ännu i hög ålder med en positiv syn på livet. Han läste dagstidningarna mycket noggrant och inom litteraturen speciellt sovjetfientlig litteratur. Det hör till saken att på hans dödsbädd fanns en del sovjetfientlig litteratur halvläst.

Daniel \textdied 31.10.1958  --  Anna -Sofia \textdied 29.08.1959. Makarna ligger begravna tillsammans med 6 av sina barn som aldrig nådde vuxen ålder.
\begin{center}
  \begin{tabular}{l l l l l l}
    Med kort liv & \textborn & \textdied & Med kort liv & \textborn & \textdied \\ \hline
    Fanny & 18.02.1897 & 08.05.1907 & Agda & 10.09.1912 & 18.10.1914 \\
    Sigrid & 08.09.1901 & 13.12.1905 & Torsten & 28.02.1918 & 27.03.1919 \\
    August & 08.07.1905 & 28.12.1905 & Linnea & 28.02.1918 & 22.03.1919 \\
    Vuxen & \textborn &  & Vuxen & \textborn &  \\ \hline
    Isak Evert & 10.09.1899 &  & Sigurd Johannes & 24.10.1909 &  \\
    Aino Elisa & 08.07.1905 &  & Gustav \jhbold{Runar} & 10.09.1912 &  \\
    Sigrid Sofia & 08.11.1907 &  &  &  &  \\
  \end{tabular}
\end{center}
I hela syskonskaran finns 3 tvillingpar.


\jhoccupant{Jungar}{Isak \& Kajsa}{1875-1896}
Isak Johansson, \textborn 11.12.1837, gift med Kajsa Johansdr., \textborn 10.08.1833, hade 1875 övertagit hemmanet på Jungar om 5/32 mtl av Johan Danielsson och hans hustru Sanna. Dessa kvarblev på sytning. Isak Johansson benämns ``sexman'' d.v.s. en av 6 förtroendemän i socknen som skulle se till att sockenstämmans beslut verkställdes. Isak kallades också ``Luthergubben'' på grund av sin religiositet. Kajsa dog 24.01.1902. Med henne fick han barnen,
\begin{jhchildren}
  \item \jhperson{Johannes}{19.03.1867}{}
  \item \jhperson{\jhbold{Daniel}}{13.02.1875}{}
  \item \jhperson{Isak}{19.07.1879}{}
\end{jhchildren}
Isak gifte om sig med Anna Sanna Joh.dr Lassila, \textborn 25.08.1841. De vigdes den 14.09.1902 och redan 16.10 dog hon. Han gifte sig för 3:e gången med Greta Andersdotter Granqvist från Munsala, \textborn 12.03.1848. De vigdes 20.03.1904. Isak avled den 12.06.1926 och Greta flyttade därefter den 08.08.1926 tillbaka till Munsala.
