%%%
% [chapter] Jungar by
%
\jhchapter{Jungar by}

%%%
% [section] Jungar-namnet
%
\jhsection{Jungar-namnet}

Om man i öppna nätbaserade kunskapskällor slår in ordet Jungar, befinner man sig plötsligt i Nepal i Rolpa-distriktet eller  i västra inre Mongoliet vid stranden av Gula floden. Bägge alternativen ges med exakta koordinater.

Så långt borta skall vi emellertid inte söka namnets ursprung. Varifrån namnet härstammar är likväl höljt i dunkel och flera alternativa förslag finns, alla mer eller mindre sannolika. Ralf Karsten menar i sin bok ``Svensk Bygd i Österbotten II'', att namnet Jungar har ett okänt ursprung. Kan det kanske ha kommit från ``jungare'', som var benämningen på en förskärarkniv eller av ``junkhärra'', ``junkar'', d.v.s. ett begrepp för ung ädling?

Vad man i alla fall tror sig veta är att namnet Jungar i äldre tid endast användes som hemmansnamn eller släktnamn. Det skrevs då Jungaren och det dyker bland annat upp i skrift i en tingsnotis från ett Pedersöreting 13-14 augusti 1606: ``Fältes Marcus Hansson i Jungaren Saak 3 mr för åurkan Han hafr giordt på Michel Olofssons eng i Jäpu''.

På Jungar hemman om 1 mantal var husbonden vid den tiden Hans Larsson (1582--\allowbreak 1607), varför det måste ha varit hans son Marcus Hansson som senare tillträdde hemmanet (1608--\allowbreak 1627) som hade stått för nämnda åverkan. Michel Olofsson som nämns som kärande i målet var hemmansägare över Tollikko omfattande 1 mantal (1600--\allowbreak 1631).

Namnet har skrivits också på annat sätt, vilket framgår av att det år 1773 i kammararkivet i Stockholm skrevs ``Ljungo''. Karsten och också andra forskare föreslår att ändelsen -en kan ha sitt ursprung i ändelsen -vin (kort i-ljud), som enligt nordisk eller norsk tradition är en benämning för en gräsrik slätt. Professor Lars Huldén anser dock i likhet med rikssvenska handböcker att belägg för denna ändelse saknas i Finland. Vi får alltså nöja oss med att vi inte vet med säkerhet.

Även om det inte dyker upp i skrift förrän på 1500--\allowbreak 1600-talen, utesluter inte detta att namnet funnits, låt vara i talad form. Vi kan tro att det följt med ända sedan de första personerna för första gången slog sig ner just här.

%%%
% [section] Jungar by
%
\jhsection{Jungar by}

Jungar by är en typisk Österbottnisk s.k. bandby (se karta). Det märks kanske bäst av den äldsta översiktskarta vi hittat från år 1650/-51, en tid då drottning Kristina, dotter till Gustav II Adolf ännu regerade. Gårdarna är antecknade som ett pärlband längs älvens stränder och så är det i långa stycken än idag.

\jhpic{Nykarleby 1650.jpg}{Nykarleby 1650}

Som begrepp uppträder Jungar by redan i 1600-talets tingsprotokoll, men som en administrativ enhet framträder den först efter en kejserlig förordning av den 6 febr. 1865 när den kyrkliga och kommunala förvaltningen skiljdes åt. Den 24 jan. 1867 gav den kejserliga senaten tillstånd att Jeppo bildade ett eget kapell under Nykarleby församling. De nya byarna med nya gränsdragningar i förhållande både till Ytterjeppo (1867), Heikkilä och Vuokoski (1859), blev nu Överjeppo längs i huvudsak västra sidan av ån, Jungar längs i huvudsak östra sidan av ån och Lassila på tvärs i norr, gränsande till Ytterjeppo.

Byindelningen kan ses på denna karta från 1934 och har bibehållits också efter det att Jeppo kommun gått samman med Munsala, Nykarleby Landskommun och Nykarleby stad år 1975:

\jhpic{Smulters karta 1934.jpeg}{Smulters karta 1934}

Till Jungar by hörde nu gårdsgrupperna, börjande nedströms: Skog, Romar, Silvast, Fors, Grötas, Böös, Jungar, Ruotsala, Mietala, Gunnar, Tollikko och Jungarå (Krogen).

Befolkningsmängden har naturligtvis varierat under tiden. År 1750 bodde i dessa gårdsgrupper 117 personer. År 1800 var deras antal ca 420 för att ytterligare 50 år senare uppgå till 765 personer. Kring sekelskiftet år 1900 har befolkningen i Jungar by kanske nått sin topp när här är registrerade 1276 personer. Så många bodde inte i byn. Siffrorna innehåller också emigranter som fortfarande fanns inskrivna i kyrkböckerna och utgjorde för hela socknens del flera hundra personer. I Jeppo socken fanns detta år registrerade hela 2794 personer, men i tidningsnotiser från den tiden uppskattas det hemmavarande antalet till ca 2400.

Fördelningen av antalet personer mellan de olika gårdsgrupperna i Jungar by har varierat kraftigt genom tiden och kunskaper och hågkomster om människor, deras boplatser och förflyttningar har sannerligen varit en utmaning att kartlägga. Vi måste erkänna att det inte till alla delar ens varit möjligt.

Fördelningen av befolkningen mellan gårdsgrupperna med 50 års intervaller har varit:

\begin{center}
  \begin{tabular}{lcccc}
    \hline
    Hemman & 1750 & 1800 & 1850 & 1900 \\ \hline
    Skog & ? & ? & 32 & 45 \\
    Romar & 20 & 50 & 56 & 79 \\
    Stor-Silfvast & 12 & 22 & 61 & 131 \\
    Lill-Silfvast (Fors) & 13 & 48 & 32 & 91 \\
    Grötas & 9 & 45 & 120 & 132 \\
    Böös & ? & 45 & 73 & 92 \\
    Jungar & 14 & 30 & 67 & 141 \\
    Ruotsala & 15 & 36 & 58 & 119 \\
    Mietala & 12 & 49 & 78 & 113 \\
    Gunnar & 9 & 32 & 41 & 68 \\
    Tollikko & 7 & 32 & 97 & 69 \\
    Jungarå (Krogen) & 6 & 32 & 50 & 91 \\
    Löst folk & . & . & . & ca 105 \\ \hline
    Totalt & 117 & 421 & 765 & 1276 \\
    \hline
  \end{tabular}
\end{center}

%%%
% [section] Folket
%
\jhsection{Folket}

Exakt när människor först bosatte sig i dessa trakter vet vi inte, men vi vet att de vistats här under förhistorisk tid utan att för den skull blivit bofasta. När de väl blivit bofasta längs älven och börjat odla marken, har också berättelser om varifrån de kommit och vad de upplevt traderats och berättats genom släktled efter släktled.

Dessa sägner, för naturligtvis är det sägner, har glömts bort och nya har kommit till. Ännu för 100 år sedan fördes sådana sägner vidare mellan människorna när man under vinterkvällarna samlades i någons stuga eller när man under det gemensamma arbetet tog en matrast i arbetet på åker och äng. Idag har de glömts bort därför att det sociala umgänget ändrat karaktär och så många andra, kanske mer spännande impulser från en allt större värld, trängt sig på.
